\documentclass[14pt,a4paper,report]{ncc}
\usepackage[a4paper, mag=1000, left=2.5cm, right=1cm, top=2cm, bottom=2cm, headsep=0.7cm, footskip=1cm]{geometry}
\usepackage[utf8]{inputenc}
\usepackage[english,russian]{babel}
\usepackage{indentfirst}
\usepackage[dvipsnames]{xcolor}
\usepackage[colorlinks]{hyperref}
\usepackage{listings} 
\usepackage{caption}
\usepackage{itmo}
\DeclareCaptionFont{white}{\color{white}} 
\DeclareCaptionFormat{listing}{\colorbox{gray}{\parbox{\textwidth}{#1#2#3}}}
\captionsetup[lstlisting]{format=listing,labelfont=white,textfont=white}
\lstset{% Собственно настройки вида листинга
inputencoding=utf8, extendedchars=\true, keepspaces = true, % поддержка кириллицы и пробелов в комментариях
language=Pascal,            % выбор языка для подсветки (здесь это Pascal)
basicstyle=\small\sffamily, % размер и начертание шрифта для подсветки кода
numbers=left,               % где поставить нумерацию строк (слева\справа)
numberstyle=\tiny,          % размер шрифта для номеров строк
stepnumber=1,               % размер шага между двумя номерами строк
numbersep=5pt,              % как далеко отстоят номера строк от подсвечиваемого кода
backgroundcolor=\color{white}, % цвет фона подсветки - используем \usepackage{color}
showspaces=false,           % показывать или нет пробелы специальными отступами
showstringspaces=false,     % показывать или нет пробелы в строках
showtabs=false,             % показывать или нет табуляцию в строках
frame=single,               % рисовать рамку вокруг кода
tabsize=2,                  % размер табуляции по умолчанию равен 2 пробелам
captionpos=t,               % позиция заголовка вверху [t] или внизу [b] 
breaklines=true,            % автоматически переносить строки (да\нет)
breakatwhitespace=false,    % переносить строки только если есть пробел
escapeinside={\%*}{*)}      % если нужно добавить комментарии в коде
}

\begin{document}
% Переоформление некоторых стандартных названий
\renewcommand{\chaptername}{Лабораторная работа}
\def\contentsname{Содержание}
\def\student{Смирнов М.Г.}
\def\group{P3164}
\def\teacher{ФИО преподавателя}
\def\worknumber{1}
\def\studyfield{Название предмета}
\def\theme{Название темы}

% Оформление титульного листа
\maketitle

% Содержание
\tableofcontents
\newpage

%\section{Изучение картины эквипотенциальных поверхностей и силовых линий электростатического поля с помощью электролитической ванны}
%\input{lab2} Отчёт к каждой работе оформляется в отдельном файле

% Список литературы
% Для отчёта он не обязателен
\begin{thebibliography}{9}

\bibitem{latex:b1} 
Роженко~А.~И. Искусство верстки в \LaTeX'е.  Новосибирск: ИВМиМГ СО~РАН, 2005.

\bibitem{latex:b2}
Балдин~Е.~М. Компьютерная типография \LaTeX.  СПб.: БХВ-Петербург, 2008.

\bibitem{latex:b3}
Котельников~И.~А., Чеботарев~П.~З. \LaTeX\ по русски.
Новосибирск: Сибирский хронограф, 2004. 

\end{thebibliography}

\end{document}