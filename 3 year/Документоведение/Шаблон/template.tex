%!TEX TS-program = xelatex

% Этот шаблон документа разработан в 2014 году
% Данилом Фёдоровых (danil@fedorovykh.ru) 
% для использования в курсе 
% <<Документы и презентации в \LaTeX>>, записанном НИУ ВШЭ
% для Coursera.org: http://coursera.org/course/latex .
% Исходная версия шаблона --- 
% https://www.writelatex.com/coursera/latex/5.2.2

\documentclass[14pt]{report}

%%% Работа с русским языком
\usepackage[english,russian]{babel}   %% загружает пакет многоязыковой вёрстки
\usepackage{fontspec}      %% подготавливает загрузку шрифтов Open Type, True Type и др.
\defaultfontfeatures{Ligatures={TeX},Renderer=Basic}  %% свойства шрифтов по умолчанию
\setmainfont[Ligatures={TeX,Historic}]{Times New Roman} %% задаёт основной шрифт документа
\setsansfont{Comic Sans MS}                    %% задаёт шрифт без засечек
\setmonofont{Courier New}

%%% Страница
\usepackage{extsizes} % Возможность сделать 14-й шрифт

\usepackage{geometry}
\geometry{
	paperwidth=210mm,
	paperheight=297mm,
	left=20mm,
	top=3mm,
	right=10mm,
	bottom=3mm
}

%\usepackage{setspace} % Интерлиньяж
%\onehalfspacing % Интерлиньяж 1.5
%\doublespacing % Интерлиньяж 2
%\singlespacing % Интерлиньяж 1

\usepackage{minibox}
\usepackage{graphicx}
\usepackage{lipsum}
\usepackage[absolute]{textpos}
\usepackage{tikz}

\begin{document} % конец преамбулы, начало документа
	\noindent{
		\begin{tikzpicture}
		\draw(0,0)[dashed] rectangle +(11, 1.7);
		\draw(11,0)[dashed] rectangle +(3, 1.7);
		\draw(14,0)[dashed] rectangle +(4, 1.7);
		\draw(0,1.7)[dashed] rectangle +(18, 2.2);
		\draw(0,3.9)[dashed] rectangle +(18, 2);
		\draw(0,5.9)[dashed] rectangle +(18, 12);
		\draw(0, 17.9)[dashed] rectangle +(18, 1.8);
		
		\draw(0, 19.7)[dashed] rectangle +(7.5, 1.8);
		\draw(0, 21.5)[dashed] rectangle +(7.5, 2.2);
		\draw(0, 23.7)[dashed] rectangle +(7.5, 3.4);
		
		\draw(10.5, 19.7)[dashed] rectangle +(7.5, 7.4);
		
		\draw(0, 27.1)[dashed] rectangle +(18, 1.7);
		
		\draw(0,0) rectangle +(18, 28.8);
		
		
		\end{tikzpicture}
	}
\end{document} % конец документа

