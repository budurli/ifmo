\chapter{Введение}

\section{Цель занятия}

Целью практического занятия является приобретение практических навыков разработки и оформления организационно-распорядительных документов внутри предприятия.

\section{Задание}
\begin{enumerate}

\item  Создать шаблоны расположения реквизитов и границ зон для углового и продольного бланков.
\item  Составить комплект ОРД на угловых бланках по ГОСТ Р 7.0.97-2016 в соответствии с заданной ситуацией (3-5 документов)
\item  Оформить отчет по практическому занятию и представить в печатном виде в формате, предусмотренном шаблоном отчета по практическому занятию
\end{enumerate}

\section{Ситуация №4}

Во время инвентаризации библиотеки Университета ИТМО обнаружена недостача. По результатам заседания инвентаризационной комиссии и с учетом объяснительной записки заведующей библиотекой, подготовлен распорядительный документ о возмещении недостачи.


\includepdf[pages=-]{src/template.pdf}

\chapter*{Приложение}

\begin{enumerate}
	\item Какие типы документов входят в систему Организационно- распорядительная документация и их назначение?
	
	К ОРД относятся организационно-правовые документы (уставы, положения, инструкции, должностные инструкции, правила, регламенты и др.), распорядительные документы (указы, постановления, распоряжения, приказы, указания), информационно-справочные документы (переписка, докладные и служебные записки, справки, сводки, акты и др.), договорные документы (договоры, контракты, соглашения, протоколы разногласий, протоколы согласования разногласий и др.).

\item ГОСТ Р 7.0.97-2016 является документом рекомендательного характера?

Нет, он обязателен.

\item Какое количество реквизитов определено в ГОСТ Р 7.0.97-2016?

30

\item Какие реквизиты должны быть обязательно указаны во всех внутренних документах?

\item В каких реквизитах инициалы указывают перед фамилией, а в каких после (в
соответствии с ГОСТ Р 7.0.97-2016)?

\item Какие части включает текст приказа? Что представляют собой эти части?

\item Какие требования предъявляют к формату бланков организационно-
распорядительных документов?

 Документы могут создаваться на бумажном носителе и в электронной форме с соблюдением установленных правил оформления документов.
 
 При создании документа на двух и более страницах вторую и последующие страницы нуме­ руют.
 Номера страниц проставляются посередине верхнего поля документа на расстоянии не менее 10 мм от верхнего края листа.
 
 Допускается создание документов на лицевой и оборотной сторонах листа. При двустороннем создании документов ширина левого поля на лицевой стороне листа и правого поля на оборотной сто­роне листа должны быть равны.

\end{enumerate}