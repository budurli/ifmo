%!TEX TS-program = xelatex

% Этот шаблон документа разработан в 2014 году
% Данилом Фёдоровых (danil@fedorovykh.ru) 
% для использования в курсе 
% <<Документы и презентации в \LaTeX>>, записанном НИУ ВШЭ
% для Coursera.org: http://coursera.org/course/latex .
% Исходная версия шаблона --- 
% https://www.writelatex.com/coursera/latex/5.2.2

\documentclass[a4paper,12pt]{article}

%%% Работа с русским языком
\usepackage[english,russian]{babel}   %% загружает пакет многоязыковой вёрстки
\usepackage{fontspec}      %% подготавливает загрузку шрифтов Open Type, True Type и др.
\defaultfontfeatures{Ligatures={TeX},Renderer=Basic}  %% свойства шрифтов по умолчанию
\setmainfont[Ligatures={TeX,Historic}]{Times New Roman} %% задаёт основной шрифт документа
\setsansfont{Comic Sans MS}                    %% задаёт шрифт без засечек
\setmonofont{Courier New}


\renewcommand{\epsilon}{\ensuremath{\varepsilon}}
\renewcommand{\phi}{\ensuremath{\varphi}}
\renewcommand{\kappa}{\ensuremath{\varkappa}}
\renewcommand{\le}{\ensuremath{\leqslant}}
\renewcommand{\leq}{\ensuremath{\leqslant}}
\renewcommand{\ge}{\ensuremath{\geqslant}}
\renewcommand{\geq}{\ensuremath{\geqslant}}
\renewcommand{\emptyset}{\varnothing}

%%% Дополнительная работа с математикой
\usepackage{amsmath,amsfonts,amssymb,amsthm,mathtools} % AMS
\usepackage{icomma} % "Умная" запятая: $0,2$ --- число, $0, 2$ --- перечисление

%% Номера формул
%\mathtoolsset{showonlyrefs=true} % Показывать номера только у тех формул, на которые есть \eqref{} в тексте.
%\usepackage{leqno} % Нумерация формул слева


%%% Страница
\usepackage{extsizes} % Возможность сделать 14-й шрифт
\usepackage{geometry} % Простой способ задавать поля
	\geometry{top=25mm}
	\geometry{bottom=35mm}
	\geometry{left=35mm}
	\geometry{right=20mm}
 %
%\usepackage{fancyhdr} % Колонтитулы
% 	\pagestyle{fancy}
 	%\renewcommand{\headrulewidth}{0pt}  % Толщина линейки, отчеркивающей верхний колонтитул
% 	\lfoot{Нижний левый}
% 	\rfoot{Нижний правый}
% 	\rhead{Верхний правый}
% 	\chead{Верхний в центре}
% 	\lhead{Верхний левый}
%	\cfoot{Нижний в центре} % По умолчанию здесь номер страницы

\usepackage{setspace} % Интерлиньяж
%\onehalfspacing % Интерлиньяж 1.5
%\doublespacing % Интерлиньяж 2
%\singlespacing % Интерлиньяж 1

\usepackage{lastpage} % Узнать, сколько всего страниц в документе.

\usepackage{soul} % Модификаторы начертания

\usepackage{hyperref}
\usepackage[usenames,dvipsnames,svgnames,table,rgb]{xcolor}
\hypersetup{				% Гиперссылки
    unicode=true,           % русские буквы в раздела PDF
    pdftitle={Заголовок},   % Заголовок
    pdfauthor={Автор},      % Автор
    pdfsubject={Тема},      % Тема
    pdfcreator={Создатель}, % Создатель
    pdfproducer={Производитель}, % Производитель
    pdfkeywords={keyword1} {key2} {key3}, % Ключевые слова
    colorlinks=true,       	% false: ссылки в рамках; true: цветные ссылки
    linkcolor=red,          % внутренние ссылки
    citecolor=black,        % на библиографию
    filecolor=magenta,      % на файлы
    urlcolor=cyan           % на URL
}

\usepackage{csquotes} % Еще инструменты для ссылок

%\usepackage[style=authoryear,maxcitenames=2,backend=biber,sorting=nty]{biblatex}

\usepackage{multicol} % Несколько колонок

\usepackage{tikz} % Работа с графикой
\usepackage{pgfplots}
\usepackage{pgfplotstable}

\usepackage{titlesec}
\usepackage{forest}

\usepackage{multicol}
\setlength{\columnsep}{3cm}
\begin{document} % конец преамбулы, начало документа
	\begin{titlepage}
		\begin{center}
			\textsc{
				\fontsize{12pt}{14pt}\selectfont
				Федеральное государственное автономное образовательное\\
				учреждение высшего образования\\
				<<Санкт-Петербургский национальный исследовательский\\
				университет информационных технологий, механики и оптики>>\\
				Факультет безопасности информационных технологий\\
				Кафедра проектирования и безопасности компьютерных систем\\}
			
			\vfill
			
			\textbf{Дисциплина}\\
			<<Защита и обработка конфиденциальных документов>>\\
			
			\vfill
			
			\textbf{Отчёт по лабораторной работе №1}\\
			<<Организация службы делопроизводства>>\\
			
		\end{center}
		
		\vfill
		
		\begin{flushright}
			\textbf{Выполнил}: \\
			студент группы N3364\\
			\rule{10em}{.1pt} /Смирнов М.Г./\\
			\vfill
			\textbf{Проверил}:\\
			Гавричков А.В.\\
			
			\rule{10em}{.1pt} /А.В. Гаричков/\\
			
			\vfill
			
			Отметка о выполнении \rule{10em}{.1pt}
			
			
			
			
			
		\end{flushright}
		\vfill
		\begin{center}
			Санкт-Петербург, 2018 г.
		\end{center}
	\end{titlepage}
\pagestyle{empty}
	
\chapter{Положение о службе делопроизводства}

\begin{multicols}{2}
	ООО <<Аглая>>\\
	\vfill
	ПОЛОЖЕНИЕ 10.00.1 № 1\\ 
	Санкт-Петербург\\
	\vfill\null
	\columnbreak
	УТВЕРЖДАЮ \\
	\\
		/\rule{10em}{1pt}\ / \\
\end{multicols}

\vfill
\centering{СТРУКТУРА ТЕКСТА}

\begin{enumerate}
	\item Общие положения
	\item Основные задачи
	\item Функции
	\item Права и обязанности
	\item Ответственность
	\item Взаимоотношения
\end{enumerate}

\vfill

\noindent{
\begin{multicols}{2}
	Руководитель структурного подразделения\\
	\vfill
	С положением ознакомлены:
	\vfill\null
	\columnbreak
	/\rule{10em}{1pt}\ / \\
	
		/\rule{10em}{1pt}\ /
\end{multicols}
}
	
	
\newpage
\pagestyle{plain}
\titleformat{\section}[block]{\Large\bfseries\filcenter}{\thesection}{1em}{}
\renewcommand*{\theenumi}{\thesection.\arabic{enumi}}
\renewcommand*{\theenumii}{\theenumi.\arabic{enumii}}

\section{Общие положения}
\begin{enumerate}
	\item Отдел по вопросам государственной службы, кадров и делопроизводства (далее - Отдел) является структурным подразделением ООО <<Аглая>>.
	\item Отдел создан на основании приказа руководителя №1 от <<10>> октября 2018 г.
	\item Работники отдела назначаются и освобождаются от должности на основании решения руководителя по представлению начальника отдела.
	\item Сотрудники отдела должны знать и руководствоваться в своей работе следующим:
	\begin{itemize}
		\item Конституция Российской Федерации;
		\item постановления, распоряжения, приказы, другие руководящие и нормативные документы вышестоящих органов, касающиеся организации делопроизводства;
		\item Единая государственная система делопроизводства <*>; стандарты унифицированной системы организационно-распорядительной документации;
		\item структура предприятия, учреждения, организации; организация делопроизводства; схемы документооборота;
		\item порядок составления номенклатуры дел, описей дел постоянного и временного хранения, установленной отчетности; сроки и порядок сдачи дел в архив;
		\item основы трудового законодательства;
		\item правила и нормы охраны труда;
		\item локальные акты предприятия.
	\end{itemize}
\item В отделе должны быть документы и материалы по следующим вопросам:
\begin{itemize}
	\item методические материалы по организации делопроизводства, схемы документооборота;
	\item порядок составления номенклатуры дел, описей дел постоянного и временного хранения, установленной отчетности;
	\item сроки и порядок сдачи дел в архив;
	\item Единая государственная система делопроизводства;
	\item стандарты унифицированной системы организационно-распорядительной документации;
	\item структура предприятия и его подразделений;
	\item системы организации контроля за исполнением документов;
	\item методы эффективного применения оргтехники и других технических средств управленческого труда;
	\item правила внутреннего трудового распорядка;
	\item правила и нормы охраны труда, пожарной безопасности.
\end{itemize}
\end{enumerate}

\section{Основные задачи}
\begin{enumerate}
	\item Основными задачами отдела являются:
	\begin{itemize}
		\item обеспечение организации делопроизводства на предприятии в соответствии с типовой инструкцией по делопроизводству и работе архива;
		\item организация и обеспечение функционирования в структурных подразделениях предприятия Единой системы делопроизводства;
		\item организация и обеспечение единого порядка работы с документами;
		\item совершенствование форм и методов делопроизводства;
		\item осуществление контроля за своевременным прохождением, исполнением и качественным оформлением документов в структурных подразделениях предприятия, анализ исполнительской дисциплины;
		\item организация работы с исходящей и входящей корреспонденцией;
		\item систематизация и хранение документов текущего архива;
		\item учет объема документооборота;
		\item разработка, внедрение новых технологических процессов работы с документами и документной информацией на основе использования средств организационной и вычислительной техники, в том числе упорядочение состава документов и информационных показателей, сокращение их количества и оптимизация документопотоков;
		\item организация методической помощи в работе с документами в структурных подразделениях предприятия;
		\item подготовка документов к передаче в архив.
	\end{itemize}
\end{enumerate}

\section{Функции}

В соответствии с возложенными на него задачами отдел осуществляет следующие функции:

\begin{enumerate}
	\item Организует в целом делопроизводство на предприятии.
	
	\item Разрабатывает, внедряет новые технологические процессы работы с документами и документной информацией, способствующие сокращению сроков прохождения и исполнения документов; принимает меры по упорядочению состава документов, оптимизации документопотоков и документооборота в целом.
	
	\item Принимает участие в постановке задач, проектировании, эксплуатации и совершенствовании (в части информационного обеспечения) автоматизированных информационных систем, а также новейших информационных технологий (в том числе безбумажных), применяемых на предприятии.
	
	\item Обеспечивает своевременное рассмотрение документов, представляемых для доклада руководству предприятия.
	
	\item Проверяет качество оформления документов, представляемых на подпись руководству предприятия.
	
	\item Осуществляет экспедиционную обработку, прием, регистрацию, учет, хранение, доставку и рассылку корреспонденции (входящей, исходящей, внутренней), в том числе переданной по специальным средствам связи, ведет справочную работу по ней.
	
	\item Обеспечивает соответствующий режим доступа к документам и использования информации, содержащейся в них.
	
	\item Организует и осуществляет машинописное (или с применением средств вычислительной техники) изготовление, копирование, оперативное размножение документов.
	
	\item Осуществляет методическое руководство и контроль за организацией документационного обеспечения в структурных подразделениях предприятия.
	
	\item Ведет учет объема документооборота.
	
	\item Организует работу по приему руководством предприятия посетителей по личным вопросам, а также делопроизводство по обращениям граждан в соответствии с типовой инструкцией по делопроизводству и работе архива на предприятии.
	
	\item Принимает участие в организации технического обслуживания созываемых руководством предприятия совещаний и заседаний.
	
	\item Организует и проводит мероприятия по повышению квалификации должностных лиц, занятых документационным обеспечением деятельности предприятия.
	
	\item Осуществляет методическое руководство деятельностью других структурных подразделений по вопросам делопроизводства.
	
	\item Ведет в рамках своей компетенции делопроизводство, осуществляет формирование и отправление/получение корреспонденции и другой информации по электронным каналам связи.
	
	\item Организует ведение нормативно-справочной информации, относящейся к функциям отдела.
	
	\item Обеспечивает в пределах своей компетенции защиту сведений, составляющих государственную тайну, и иных сведений ограниченного распространения.
	
	\item Осуществляет в соответствии с законодательством Российской Федерации работу по комплектованию, хранению, учету и использованию архивных документов, образовавшихся в ходе деятельности отдела.
\end{enumerate}

Возложение на отдел функций, не относящихся к делопроизводству, не допускается.

\section{Права и обязанности}

Отдел делопроизводства для решения возложенных на него задач имеет право:

\begin{enumerate}
	\item осуществлять проверку организации делопроизводства в структурных подразделениях предприятия, о результатах проверок докладывать руководству;
	
	\item запрашивать и получать в установленном порядке от структурных подразделений предприятия необходимые документы для выполнения возложенных на отдел функций;
	
	\item запрашивать от руководства и структурных подразделений предприятия необходимые документы по вопросам, связанным с реализацией возложенных на отдел прав и обязанностей;
	
	\item требовать от руководителей структурных подразделений предприятия выполнения установленных правил работы с документами, входящей и исходящей корреспонденцией, своевременного исполнения распоряжений руководства предприятия;
	
	\item вносить руководству предприятия предложения по совершенствованию форм и методов работы с документами в структурных подразделениях и администрации предприятия;
	
	\item возвращать исполнителям на доработку ответы, подготовленные с нарушением установленных требований;
	
	\item привлекать специалистов структурных подразделений предприятия к подготовке проектов методических документов по вопросам документационного обеспечения, делопроизводства и архива.
\end{enumerate}

\section{Ответственность}

\begin{enumerate}
	\item Всю полноту ответственности за качество и своевременность выполнения возложенных Положением на отдел задач и функций несет начальник отдела.
	
	\item Степень ответственности других работников устанавливается должностными инструкциями.
	
	\item Начальник и другие сотрудники отдела несут персональную ответственность за соответствие оформляемых ими документов и операций с корреспонденцией законодательству Российской Федерации.
\end{enumerate}

\section{Взаимоотношения}

\begin{enumerate}
	\item  В процессе производственной деятельности предприятия отдел постоянно взаимодействует со следующими структурными подразделениями:
	\begin{itemize}
		\item Отдел разработки
		\item Юридический отдел
	\end{itemize}
	\item По вопросам, относящимся к его компетенции, отдел оказывает содействие всем подразделениям предприятия.
\end{enumerate}

\newpage
\pagestyle{plain}
\titleformat{\section}[block]{\Large\bfseries\filcenter}{\thesection}{1em}{}
\renewcommand*{\theenumi}{\thesection.\arabic{enumi}}
\renewcommand*{\theenumii}{\theenumi.\arabic{enumii}}

\setcounter{section}{0}

\flushright{
	Утверждаю\\
	Генеральный директор \\
			/\rule{10em}{1pt}\ / \\
}

\vfill

\centering{
	ДОЛЖНОСТНАЯ ИНСТРУКЦИЯ\\
	архивариуса отдела делопроизводства 
}

\vfill

\section{Общие положения}
\begin{enumerate}
	\item Архивариус относится к категории технических специалистов.
	\item На должность архивариуса назначается лицо, имеющее начальное профессиональное образование без предъявления требований к стажу работы или среднее (полное) общее образование и специальную подготовку по установленной программе без предъявления требований к стажу работы.
	\item Назначение на должность архивариуса и освобождение от нее производится приказом генерального директора по представлению руководителя отдела делопроизводства.
	\item Архивариус должен знать
	\begin{itemize}
		\item Нормативные правовые акты, положения, инструкции, другие руководящие материалы и документы по ведению архивного дела на предприятии.
		\item Порядок приема и сдачи документов в архив, их хранение и пользование ими.
		\item Единую государственную систему делопроизводства.
		\item Порядок составления описаний документов постоянного и временного хранения и актов об уничтожении документов.
		\item Порядок оформления дел и их подготовки к хранению и использованию.
		\item Порядок ведения учета и составления отчетности.
		\item Структуру предприятия.
		\item Основы организации труда.
		\item Правила эксплуатации технических средств.
		\item Законодательство о труде.
		\item Правила внутреннего трудового распорядка.
		\item Правила и нормы охраны труда, противопожарной защиты.
	\end{itemize}

\newpage
\item Архивариус в своей деятельности руководствуется:
\begin{itemize}
	\item Положением об отделе делопроизводство
	\item Настоящей должностной инструкцией
\end{itemize}

\item Архивариус подчиняется непосредственно руководителю отдела делопроизводства
\end{enumerate}

\section{Должностные обязанности}

\begin{enumerate}
	\item Основными задачами архивариуса являются:
	\begin{itemize}
		\item Обеспечение отбора, упорядочения, комплектования, использования, сохранности принимаемых в архив документов (в т.ч. законченных делопроизводством документов практического назначения).
		\item Создание справочного аппарата к ним
		\item Осуществление контроля за формированием и оформлением дел в делопроизводстве подразделений организации
	\end{itemize}
	\item Для выполнения вышеуказанных задач архивариус:
	\begin{itemize}
		\item Организует хранение и обеспечивает сохранность документов, поступивших в архив.
		\item Принимает и регистрирует поступившие на хранение от структурных подразделений документы, законченные делопроизводством.
		\item Участвует в разработке номенклатуры дел, проверяет правильность их формирования и оформления при передаче в архив.
		\item В соответствии с действующим законодательством шифрует единицы хранения, систематизирует и размещает дела, ведет их учет.
		\item Подготавливает сводные описи единиц постоянного и временного сроков хранения, а также акты для передачи документов на государственное хранение, на списание и уничтожение материалов, сроки хранения которых истекли.
		\item Ведет работу по созданию справочного аппарата по документам, обеспечивает удобный и быстрый их поиск.
		\item Участвует в работе по экспертизе научной и практической ценности архивных документов.
		\item Следит за состоянием документов, своевременностью их восстановления, соблюдением в помещениях архива условий, необходимых для обеспечения их сохранности.
		\item Контролирует соблюдение правил противопожарной защиты в помещении архива.
		\item Выдает в соответствии с поступающими запросами архивные копии и документы, составляет необходимые справки на основе сведений, имеющихся в документах архива, подготавливает данные для составления отчетности о работе архива
		\item Участвует в работе по экспертизе ценности архивных документов.
		\item Принимает необходимые меры по использованию в работе современных технических средств.
	\end{itemize}

\end{enumerate}

\section{Права}
\flushleft{Архивариус имеет право:}
\begin{enumerate}
	\item Знакомиться с проектами решений руководства предприятия, касающимися его деятельности.
	\item Вносить предложения по совершенствованию работы, связанной с предусмотренными настоящей инструкцией обязанностями.
	\item . В пределах своей компетенции сообщать руководитею отдела о всех недостатках в деятельности предприятия (структурных подразделений, отдельных работников), выявленных в процессе исполнения своих должностных прав и обязанностей и вносить предложения по их устранению.
	\item Запрашивать лично или по поручению руководителя отдела от руководителей подразделений предприятия и иных специалистов информацию и документы, необходимые для выполнения его должностных обязанностей.
	\item Требовать от руководства предприятия оказания содействия в исполнении обязанностей и прав, предусмотренных настоящей должностной инструкцией.
\end{enumerate}

\section{Ответственность}

\flushleft{Архивариус несет ответственность:}

\begin{enumerate}
	\item За ненадлежащее исполнение или неисполнение своих должностных обязанностей, предусмотренных настоящей должностной инструкцией - в пределах, определенных действующим трудовым законодательством Российской Федерации.
	\item За правонарушения, совершенные в процессе осуществления своей деятельности - в пределах, определенных действующим административным, уголовным и гражданским законодательством Российской Федерации
	\item За причинение материального ущерба - в пределах, определенных действующим трудовым и гражданским законодательством Российской Федерации.
\end{enumerate}

\newpage

\section*{Структура отдела делопроизводства}

\begin{forest}
	for tree={
		font=\ttfamily,
		grow'=0,
		child anchor=west,
		parent anchor=south,
		anchor=west,
		calign=first,
		edge path={
			\noexpand\path [draw, \forestoption{edge}]
			(!u.south west) +(7.5pt,0) |- node[fill,inner sep=1.25pt] {} (.child anchor)\forestoption{edge label};
		},
		before typesetting nodes={
			if n=1
			{insert before={[,phantom]}}
			{}
		},
		fit=band,
		before computing xy={l=15pt},
	}
	[Руководитель отдела делопроизводства
		[Главный специалист отдела делопроизводства
			[Ведущий специалист отдела делопроизводства]
			[Специалист отдела делопроизводства]			
			[Специалист отдела делопроизводства]						
		]
		[Архивариус]
	]
\end{forest}

\end{document} % конец документа
