\section*{Введение}
\addcontentsline{toc}{section}{Введение}%

Интероперабельность – способность двух или более информационных систем или компонентов к обмену информацией и к использованию информации, полученной в результате обмена. Интероперабельность играет значимую роль при создании систем промышленной автоматизации и их интеграции, и, наряду со свойством переносимости, является важнейшей составляющей понятия <<открытые системы>>. В настоящее время все большее внимание уделяется именно вопросам обеспечения интероперабельности для информационных систем различного масштаба (от наносистем до <<системы систем>>) и информационных систем (ИС) различных областей назначения. Причина того, что в настоящее время интероперабельность приобретает все большее значение, в первую очередь в том, что сегодня практически ни одна сфера жизни (государственное управление, здравоохранение, образование, наука, бизнес и др.) не обходится без использования информационно-коммуникационных технологий.

Поскольку именно существующие информационные системы всё больше формируют наши представления об объектах, то отсутствие интероперабельности между различными ИС ведёт к <<зашориванию>> наших знаний.

Целью данной работы будет выявление интероперабельности в существующих системах и сравнение их.