\chapter{Введение}

\section{Цель работы}

Получение навыков импортирования и экспортирования данных в БД

\section{Теоретическая информация}

Так как любая БД представляется собой структурированное хранилище данных необходимо наличие утилит для осуществления импортирование данных в БД и экспортирования для переноса в другую БД или для независимой обработки. В СУБД MySQL для этого существует несколько инструментов.

\chapter{Ход работы}



\section{Произвести экспортирование только структуры таблицы}

\lstinputlisting{src/1.in}

\lstinputlisting{src/1.out}

\section{Произвести экспортирование данных и структуры таблицы}

\lstinputlisting{src/2.in}

\lstinputlisting{src/2.out}

\section{Произвести экспортирование только данных таблицы}

\lstinputlisting{src/3.in}

\lstinputlisting{src/3.out}

\section{Импортировать данных и структуру таблицы в новую БД}

\lstinputlisting{src/4.in}

\chapter{Вывод}

В рамках данной работы были получены первичные навыки импортирования и экспортирования информации в БД.