\section{Ход работы}

\begin{table}[h]
	\centering
	\caption{Входная характеристика оптрона}
	\begin{tabular}{|c|c|}
		\hline
		$U_\text{ВХ}$, В   & $I_\text{ВХ}$ , мА   \\ \hline
		0   & 0    \\ \hline
		0.2 & 0    \\ \hline
		0.4 & 0    \\ \hline
		0.6 & 1    \\ \hline
		0.8 & 3.5  \\ \hline
		1   & 7    \\ \hline
		1.2 & 8.5  \\ \hline
		1.4 & 11.5 \\ \hline
		1.6 & 15   \\ \hline
		1.8 & 17.5 \\ \hline
		1.9 & 18.5 \\ \hline
	\end{tabular}
\end{table}

\begin{figure}[!h]
	\centering
\begin{tikzpicture}
\begin{axis}[
title={Входная характеристика оптрона},
xlabel={$U_\text{ВХ}$, В},
ylabel={$I_\text{ВХ}$, мА },
xmin=0, xmax=3,
ymin=0, ymax=20,
legend pos=north west,
ymajorgrids=true,
grid style=dashed,
]

\addplot[
color=blue,
mark=square,
smooth
]
coordinates {
(0.00,	0.00)(0.20,	0.00)(0.40,	0.00)(0.60,	1.00)(0.80,	3.50)(1.00,	7.00)(1.20,	8.50)(1.40,	11.50)(1.60,	15.00)(1.80,	17.50)(1.90,	18.50)
};

\end{axis}
\end{tikzpicture}
\end{figure}
\newpage

\begin{table}[!h]
	\centering
	\caption{Выходные характеристики оптрона}
		\begin{tabular}{|c|c|c|c|c|c|c|}
			\hline
			& $U_\text{ВЫХ}$ &  &  &  &  &  \\ \hline
			$I_\text{ВХ}$ & 0 & 4,5 & 5 & 5,5 & 6 & 6,5 \\ \hline
			0 & 0 & 0 & 0 & 0 & 2 & 2 \\ \hline
			10 & 0 & 0 & 0 & 2 &  &  \\ \hline
			20 & 0 & 0 & 1 &  &  &  \\ \hline
			30 & 0 & 0 & 2 &  &  &  \\ \hline
			40 & 0 & 0 &  &  &  &  \\ \hline
			50 & 0 & 2 &  &  &  &  \\ \hline
			60 & 0 & 4 &  &  &  &  \\ \hline
			70 & 0 &  &  &  &  &  \\ \hline
			80 & 0 &  &  &  &  &  \\ \hline
			90 & 0 &  &  &  &  &  \\ \hline
			100 & 0 &  &  &  &  &  \\ \hline
			110 & 0 &  &  &  &  &  \\ \hline
			120 & 0 &  &  &  &  &  \\ \hline
			130 & 0 &  &  &  &  &  \\ \hline
			140 & 0 &  &  &  &  &  \\ \hline
			150 & 0 &  &  &  &  &  \\ \hline
			160 & 0 &  &  &  &  &  \\ \hline
			170 & 0 &  &  &  &  &  \\ \hline
			180 & 0 &  &  &  &  &  \\ \hline
			190 & 0 &  &  &  &  &  \\ \hline
			200 & 2 &  &  &  &  &  \\ \hline
		\end{tabular}
\end{table}


\begin{figure}[!h]
	\centering
	\begin{tikzpicture}
	\begin{axis}[
	title={Выходные характеристика оптрона},
	xlabel={$U_\text{ВЫХ}$, В},
	ylabel={$I_\text{ВХ}$, мА },
	xmin=-10, xmax=70,
	ymin=-1, ymax=5,
	legend pos=north west,
	ymajorgrids=true,
	grid style=dashed,
	]
	
	\addplot
	coordinates {
		(0,0)(10,0)(20,0)(30,0)(40,0)(50,2)(0,4)
	};

	\addplot
	coordinates {
		(0,0)(10,0)(20,1)(0,2)
	};

	\addplot
	coordinates {
		(0,0)(10,1)(0,2)
	};

	
	\end{axis}
	\end{tikzpicture}
\end{figure}


\newpage

\begin{table}[!h]
	\centering
	\caption{Минимальный выходной ток}
	\begin{tabular}{|c|c|}
		\hline
		$I_\text{вх}$, мА & $I_\text{вых\_мин}$, мА \\ \hline
		2.5   & 4           \\ \hline
		5     & 4           \\ \hline
		7.5   & 6           \\ \hline
		10    & 8           \\ \hline
	\end{tabular}
\end{table}

\begin{figure}[!h]
	\centering
	\begin{tikzpicture}
	\begin{axis}[
	title={Минимальный выходной ток},
	xlabel={$I_\text{вх}$, мА},
	ylabel={$I_\text{вых\_мин}$, мА},
	xmin=0, xmax=12,
	ymin=0, ymax=12,
	legend pos=north west,
	ymajorgrids=true,
	grid style=dashed,
	]
	
	\addplot[
	color=blue,
	mark=square,
	smooth
	]
	coordinates {
		(2.5,	4)(5,	4)(7.5,	6)(10,	8)
	};
	
	\end{axis}
	\end{tikzpicture}
\end{figure}

\newpage

\begin{table}[!h]
	\centering
	\caption{Зависимость входного тока срабатывания от напряжения на
		выходе оптрона}
	\begin{tabular}{|c|c|}
		\hline
		$U_\text{вых}$, В & $I_\text{ср}$, мА \\ \hline
		80     & 5     \\ \hline
		90     & 5.5   \\ \hline
		100    & 4.5   \\ \hline
		110    & 4     \\ \hline
		120    & 3.5   \\ \hline
		160    & 3     \\ \hline
	\end{tabular}
\end{table}

\begin{figure}[!h]
	\centering
	\begin{tikzpicture}
	\begin{axis}[
	title={Зависимость входного тока срабатывания от напряжения на
		выходе оптрона},
	xlabel={$U_\text{вых}$, В},
	ylabel={$I_\text{ср}$, мА},
	xmin=70, xmax=180,
	ymin=2, ymax=6,
	legend pos=north west,
	ymajorgrids=true,
	grid style=dashed,
	]
	
	\addplot[
	color=blue,
	mark=square,
	smooth
	]
	coordinates {
(80,	5)(90,	5.5)(100,	4.5)(110,	4)(120,	3.5)(160,	3)
	};
	
	\end{axis}
	\end{tikzpicture}
\end{figure}

\[
I_\text{ВЫХ\_УД} = 8 \text{ мА}
\]

\section{Вывод}

В ходе работы произошло ознакомление с основными параметрами и характеристиками оптрона типа <<светоизлучающий диод - фототиристор>>.
Графики, полученные в ходе работы, подтверждают теоретические сведения.

