\section{Краткое описание предприятия и его структура}

EPAM Systems — американская ИТ-компания, основанная в 1993 году. Крупнейший мировой производитель заказного программного обеспечения, специалист по консалтингу, резидент Белорусского парка высоких технологий. Штаб-квартира компании расположена в Ньютауне, штат Пенсильвания, а её отделения представлены в 25 странах мира.

\subsection{История компании}

Компания EPAM была основана в 1993 году двумя одноклассниками Аркадием Добкиным и Леонидом Лознером. Название компании происходило от <<Effective Programming for America>>. Первые офисы были открыты в США и Беларуси. Позже были открыты центральный североамериканский офис в Лоренсвилле, США, штат Нью-Джерси и центральный европейский офис в Будапеште, Венгрия, а также офисы по обслуживанию клиентов в Австрии, Австралии, Армении, Болгарии, Белоруссии, Великобритании, Германии, Индии, Ирландии, Казахстане, Канаде, Китае, Мексике, Нидерландах, ОАЭ, Польше, России, Сингапуре, Украине, Чехии, Швеции, Швейцарии.

В марте 2004 года EPAM приобрела компанию Fathom Technology в Венгрии, а в сентябре 2006 VDI в России, образовав единую компанию под именем EPAM Systems со штатом сотрудников в 2200 человек.

В 2012 году компания совершает ряд приобретений на северо-американском рынке, в числе которых канадская компания Thoughtcorp и крупный поставщик услуг по разработке цифровых стратегий и организации мультиканального взаимодействия Empathy Lab.

В 2014 году EPAM приобрела китайскую ИТ-компанию Jointech (Joint Technology Development Limited), за счёт чего, как следует из пресс-релиза EPAM, расширила свои возможности в Азиатско-Тихоокеанском регионе и купила американского поставщика услуг для здравоохранения и медико-биологического сектора GGA Software Services.

В 2015 EPAM Systems поглотил американские компании: NavigationArtsruen, специализирующуюся на цифровом консалтинге и дизайне, а также Alliance Global Services, которая специализируется на выпуске ПО и решений для автоматизированного тестирования. В связи с этим приобретением руководство EPAM Systems пересмотрело прогноз по выручке в сторону увеличения, ожидая её на уровне не ниже 905 млн долларов в 2015 году против 730 млн годом ранее.

В 2016 году EPAM поглотила китайскую компанию Dextrys, со штатом в 400 сотрудников.

В 2018 году EPAM поглотила американскую компанию Continuum, со штатом более 150 человек.

25 января 2012 года объявила о начале подготовки к IPO на Нью-Йоркской фондовой бирже. Как говорилось в сообщении компании, крупнейший пакет акций EPAM Systems могут продать фонды, аффилированные с инвесткомпанией Siguler Guff \& Co: после IPO они уменьшат долю в EPAM Systems с 52,5\% до 41,2\%. Основатель и генеральный директор Аркадий Добкин рассчитывал продать 2\% акций компании. Всего акционеры EPAM Systems предполагали разместить до 14\% акций компании. Ещё 3,6\% акций допэмиссии, как планировалось, продаст сама компания, и объём размещения составит 17,7\% на \$133,2 млн.

IPO состоялось 8 февраля и было оценено аналитиками как неудачное. В ходе размещения было продано 6 млн акций (14,7\% увеличенного капитала) за \$72 млн, или \$12 за бумагу (при этом ранее EPAM объявляла ценовой коридор в \$16–18 за бумагу). 33\% проданных акций — дополнительная эмиссия. В соответствии с оценкой на IPO стоимость всей компании составила \$488 млн. По мнению экспертов, проблема оказалась не в бизнесе EPAM, а в непростой ситуации на рынках, особенно в Европе.

Через 2 года, в июне 2014, капитализация компании выросла более чем в четыре раза и составила \$2,14 млрд.

\subsection{Офис в Санкт-Петербурге}

История EPAM в Санкт-Петербурге началась на Свердловской набережной. Здесь в небольшом помещении, вмещающем не более 30 человек, стартовала работа над первыми проектами для клиентов.

В 2006 году росту петербургского офиса способствовало присоединение к EPAM компании ФОРС Санкт-Петербург. А в 2014 году EPAM объявил о приобретении компании GGA Software Services LLC, поставщика услуг в сфере научной информатики для глобальных фармацевтических компаний, производителей научного оборудования и медицинских приборов, научных издательств и медико-биологических компаний.

Это позволило значительно расширить спектр ИТ-услуг и технологических решений, предлагаемых петербургским филиалом EPAM для заказчиков по всему миру. Появилась возможность выделить в структуре офиса специализированные направления, отвечающие за различные сферы разработки программного обеспечения. Впоследствии сформировались команды настоящих профессионалов, которые детально знают свою предметную область и способны решить самые сложные задачи.

Сегодня силами петербургского офиса в компании реализуются одни из самых сложных и интересных проектов для российских и иностранных заказчиков. В Санкт-Петербурге работают свыше 1500 разработчиков, инженеров по качеству, дизайнеров, архитекторов, стратегов и экспертов по продуктам EPAM. В распоряжении сотрудников компании находятся современные и просторные офисы в центре Санкт-Петербурга.