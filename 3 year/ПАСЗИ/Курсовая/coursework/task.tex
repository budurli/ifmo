% !TeX spellcheck = ru_RU
\thispagestyle{empty}


\begin{center}
{\small САНКТ-ПЕТЕРБУРГСКИЙ НАЦИОНАЛЬНЫЙ ИССЛЕДОВАТЕЛЬСКИЙ УНИВЕРСИТЕТ ИНФОРМАЦИОННЫХ ТЕХНОЛОГИЙ, МЕХАНИКИ И ОПТИКИ}

\textbf{ЗАДАНИЕ НА КУРСОВОЙ ПРОЕКТ (РАБОТУ)}
\end{center}



Студент:  \rulefill{\FullName}

Факультет: \rulefill{\Faculty}

Кафедра: \rule{10em}{.1pt} Группа: \rulefill{\Group}

Направление:  \rulefill{\Speciality}

Руководитель: \rulefill{\Teacher}

Дисциплина: \rulefill{\Subject}

\hrulefill

Наименование темы: \rulefill{\Theme}

\hrulefill

Задание: \rulefill{Разработать методологию сравнения сканеров уязвимостей}

\rulefill{программного обеспечения и провести сравнение сканеров XSpider и Nessus}

\hrulefill

\hrulefill

Краткие методические указания: \rulefill{}

\rulefill{Иванова Н.Ю., Комарова И.Э., Малинин А.А. <<Подготовка презентаций для курсовых проектов и выпускных квалификационных работ>>}

\hrulefill

\hrulefill

Содержание пояснительной записки: \hrulefill

\rulefill{Введение}

\rulefill{1. Теоретическая часть}

\rulefill{1. 1. Данные производителей}

\rulefill{2. Практическая часть}

\rulefill{Заключение}



Рекомендуемая литература: \rulefill{ГОСТ Р 50922-2006 <<Защита информации>>;}

\rulefill{Руководящий документ <<Защита от несанкционированного доступа к}

\rulefill{информации. Термины и определения>>; }

\hrulefill{}



Руководитель: \hrulefill

Студент: \hrulefill
