\chapter{Основная часть}

\section{Норматив оборотных средств}

\[
\text{Н}_\text{о}^\text{пз}=\frac{\text{Р}_\text{пл}}{\text{Д}_\text{пл}} \cdot (\text{J}+\text{Д}_\text{н}^\text{тр}+\text{Д}_\text{н}^\text{стр})= \frac{199000}{90} \cdot (26 + 0.5 \cdot 26 + 0.25 \cdot 26)=100605.56
\]

\section{Норматив оборотных средств по незавершенному производству}
\[
\text{Н}_\text{о}^\text{нп} = \frac{\text{С}_\text{пл}}{\text{Д}_\text{пл}} \cdot \text{Т}_\text{ц} \cdot \text{К}_\text{нз}=\frac{685000}{90} \cdot 4 \cdot 0.5 = 15222.22
\]

\section{Норматив оборотных средств по элементу <<Расходы будущих периодов>>}

\[
\text{Н}_\text{о}^\text{рб}=18000
\]

\section{Норматив оборотных средств по элементу <<Готовая продукция>>}

\[
\text{Н}_\text{о}^\text{гп} =  \frac{\text{С}_\text{пл}}{\text{Д}_\text{пл}} \cdot \text{Д}_\text{н}^\text{гп}=\frac{685000}{90} \cdot 3 = 22833.34
\]

\section{Общий норматив оборотных средств}

\[
\text{Н}_\text{о}=\text{Н}_\text{о}^\text{пз}+\text{Н}_\text{о}^\text{нп}+\text{Н}_\text{о}^\text{рб}+\text{Н}_\text{о}^\text{гп}=100 605.56+152 22.22+18 000+22 833.34=156661.12
\]

\section{Оборачиваемость оборотных средств}

\[
\text{К}_\text{об}=\frac{\text{Q}_\text{рп}}{\text{Н}_\text{о}}=\frac{890500}{156661.12} \approx 5.68
\]

\section{Длительность одного оборота}

\[
\text{Т}_\text{об}=\frac{\text{Д}_\text{пл}}{\text{К}_\text{об}}=\frac{90}{5.68} \approx 15.85
\]

\section{Количество оборотов при ускорении оборачиваемости оборотных средств}

\[
\text{K}^{'}=\frac{\text{Д}_\text{пл}}{\text{Т}_\text{об}-\text{Т}}=\frac{90}{15.85-4} \approx 7.6
\]


\section{Норматив оборотных средств при ускорении оборачиваемости оборотных средств}

\[
\text{Н}_\text{о}^\text{'}=\frac{Q_\text{рп}}{\text{K}^{'}}=\frac{890500}{7.6} \approx 117171.05
\]

\section{Величина высвобождения денежных средств из оборота в результате ускорения оборачиваемости оборотных средств}

\[
\Delta \text{Н}_\text{о}=\text{Н}_\text{о}-\text{Н}_\text{о}^\text{'}=156661.12-117171.05=39490.07
\]
