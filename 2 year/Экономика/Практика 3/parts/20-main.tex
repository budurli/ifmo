\chapter{Основная часть}


\section{Технологическая трудоёмкость изделий}
\[
\text{Т}_\text{тех}=\sum_{i=1}^{4} \frac{\text{Т}_\text{н}}{\alpha_\text{i}} \cdot 100 =\left( \frac{1173}{85,1}+\frac{758}{92,7}+\frac{1557}{81,4}+\frac{3173}{67,8}\right) \cdot 100 \approx 8788.79
\]


\section{Численность основных рабочих-сдельщиков}

\[
\text{Н}_\text{орс}^\text{т}=\frac{\text{Т}_\text{тех}}{\text{Ф}_\text{п} \cdot \text{К}_\text{в}}=\frac{8788.79}{1761.6 \cdot 1.2} \approx 4.16 \approx 5
\]


\section{Численность основных рабочих-сдельщиков, определяемая по нормам выработки}

\[
\text{Н}_\text{орс}^\text{н}=\frac{\text{В}}{\text{Н}_\text{в}}=\frac{6100}{60} \approx 101.67 \approx 102
\]

\section{Численность вспомогательных рабочих-сдельщиков (ремонтное, транспортное, инструментальное подразделения)}

\[
\text{Н}_\text{орс}^\text{т}=\frac{\text{Т}_\text{вр}}{\text{Ф}_\text{п} \cdot \text{К}_\text{в}^\text{'}}=\frac{380}{1761.6 \cdot 1.15} \approx 0.19 \approx 1
\]

\section{Численность вспомогательных рабочих, занятых обслуживанием основных рабочих (электрики, наладчики, слесари-ремонтники, контролеры)}

\[
\text{Ч}_\text{вр}^\text{н}=\frac{\text{S} \cdot \text{С}}{\text{Н}_\text{о}}= \frac{890 \cdot 2}{\text{Н}_\text{о}}
\]

Наладчики оборудования:
\[
\text{Ч}_\text{вр}^\text{н} = \frac{890 \cdot 2}{10} = 178
\]

Слесари-ремонтники:
\[
\text{Ч}_\text{вр}^\text{н} = \frac{890 \cdot 2}{15} \approx 118,67 \approx 119
\]

Электрики:
\[
\text{Ч}_\text{вр}^\text{н} = \frac{890 \cdot 2}{20} = 89
\]

Контролеры:
\[
\text{Ч}_\text{вр}^\text{н} = \frac{890 \cdot 2}{50} = 35.6 \approx 36
\]

\section{Численность вспомогательных рабочих, определяемых по числу рабочих мест (стропали, кладовщики, комплектовщики)}

\[
\text{Ч}_\text{вр}^\text{м} = n \cdot C = n \cdot 2
\]

Стропали:

\[
\text{Ч}_\text{вр}^\text{м} = 10 \cdot 2 = 20
\]

Кладовщики:

\[
\text{Ч}_\text{вр}^\text{м} = 26 \cdot 2 = 52
\]

Комплектовщики:
\[
\text{Ч}_\text{вр}^\text{м} = 15 \cdot 2 = 30
\]

\section{Численность производственных рабочих (явочная)}

\[
\text{Ч}_\text{пр}^\text{яв} = \sum \text{Ч}_\text{ор} + \sum \text{Ч}_\text{вр} = 5 + 102 +1 +178 +119 +89 +36 +20 +52 +30 = 632
\]


\section{Коэффициент списочного состава}

\[
\text{К}_\text{сп}=\frac{\text{Ф}_\text{н}}{\text{Ф}_\text{п}} = \frac{2012}{1761.6} \approx 1.142
\]

\section{Списочная численность производственных рабочих}

\[
\text{Ч}_\text{пр}^\text{сп} = \text{Ч}_\text{пр} \cdot \text{К}_\text{сп} = 632 \cdot 1.142 = 721.744 \approx 722
\]

\section{Списочная численность промышленно-производственного персонала}

\[
\text{Ч}_\text{ппп}^\text{сп} = \frac{\text{Ч}_\text{пр}^\text{сп}}{d_\text{р}} \cdot 100 = \frac{722}{75} \cdot 100 \approx 962.67 \approx 963
\]

