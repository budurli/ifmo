\chapter{Введение}

Цель работы: на основе исходных данных рассчитать численность по категориям работающих и структуру промышленно-производственного персонала предприятия.

Для этого необходимо определить:
\begin{itemize}
	\item численность основных рабочих:
	\begin{enumerate}
		\item по трудоемкости выпускаемой продукции;
		\item по нормам выработки;
	\end{enumerate}
	\item численность вспомогательных рабочих:
	\begin{enumerate}
		\item по трудоемкости вспомогательных работ;
		\item по нормам обслуживания;
		\item по количеству рабочих мест;
	\end{enumerate}
	\item списочную численность промышленно-производственного персонала.
	
\end{itemize}

\begin{table}
	\caption{Исходные данные}
	\begin{tabular}{|p{10cm}|c|c|}
		\hline
		Наименование показателей & Значение & Ед. изм. \\ \hline
		Нормативная трудоемкость изделий А & 1173 & тыс. норм-ч \\ \hline
		Нормативная трудоемкость изделий Б & 758 & тыс. норм-ч \\ \hline		
		Нормативная трудоемкость изделий В & 1557 & тыс. норм-ч \\ \hline
		Нормативная трудоемкость изделий Г & 3173 & тыс. норм-ч \\ \hline
		Процент выхода годных изделий А & 85.1 & \% \\ \hline
		Процент выхода годных изделий Б & 92.7 & \% \\ \hline
		Процент выхода годных изделий В & 81.4 & \% \\ \hline
		Процент выхода годных изделий Г & 67.8 & \% \\ \hline				
		Коэффициент выполнения норм рабочими-сдельщиками & 1.2 & \\ \hline
		Объем работ, связанных с выполнением основной программы & 6100 & шт. \\ \hline
		Выработка одного рабочего & 60 & шт. \\ \hline
		Трудоемкость вспомогательных работ & 380 & тыс. нормо-ч. \\ \hline
		Коэффициент выполнения норм вспомогательными рабочими-сдельщиками & 1.15 & \\ \hline
		Количество единиц обслуживаемого оборудования & 890 & шт. \\ \hline
	\end{tabular}
\end{table}