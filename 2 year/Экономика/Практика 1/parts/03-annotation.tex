\chapter{Введение}

Целью работы является оценка использования основных фондов предприятия, для этого будут поэтапно рассчитаны некоторые показатели.

Исходные данные представлены в таблицах \ref{tbl:initial} и \ref{tbl:indicators}.

\begin{table}
	\caption{}
	\label{tbl:initial}
	\centering
	
	\begin{tabular}{|c|p{10cm}|c|c|}
		\hline
№ & Группа ОПФ & ФО & ТФ \\ \hline		
1.  &  Здания & 28642 & 35 \\ \hline
2.  &  Сооружения & 1765 & 15 \\ \hline
3.  &  Передаточные устройства & 1830 & 12 \\ \hline
4.  &  Машины и оборудование,   в  т.ч. &  &  \\ \hline
4.1. &     Силовые машины и оборудование & 1326 & 16 \\ \hline
4.2. &     Рабочие  машины и оборудование & 29420 & 8 \\ \hline
4.3. &     Измерительные и регулирующие приборы и устройства, лабораторное оборудование & 2485 & 4,5 \\ \hline
4.4. &     Вычислительная техника & 2216 & 3 \\ \hline
5.  &  Транспортные средства & 1544 & 5 \\ \hline
6.  &  Инструмент & 685 & 3 \\ \hline
7.  &  Производственный и хозяйств. инвентарь и принадлежности & 276 & 6 \\ \hline
	\end{tabular}
\end{table}

\begin{table}
	\caption{}
	\label{tbl:indicators}
	\centering
	\begin{tabular}{|c|c|}
		\hline
		Оптовая цена станка, тыс.руб. & 10,6 \\ \hline
		Год приобретения & 1995 \\ \hline
		Транспортные расходы, \% от оптовой цены & 5 \\ \hline
		Затраты на монтаж и отладку, \% от оптовой цены & 10 \\ \hline
		Стоимость ОПФ на 1 января текущего года, тыс. у.д.е. & 65 \\ \hline
		Введено ОПФ в текущем году,  тыс. у.д.е. & 12 \\ \hline
		Дата ввода & июль \\ \hline
		Выбыло ОПФ в текущем году,  тыс. у.д.е. & 7 \\ \hline
		Дата выбытия & октябрь \\ \hline
	\end{tabular}

\end{table}

