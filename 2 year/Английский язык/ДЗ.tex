\documentclass[a4paper,12pt]{article}

\usepackage[T2A]{fontenc}
\usepackage[utf8]{inputenc}
\usepackage[english,russian]{babel}
\usepackage{indentfirst}
\usepackage{graphicx}
\usepackage{wrapfig}
\usepackage{amsmath}
\usepackage{lscape}
\usepackage{fancyhdr}

\usepackage{caption}
\begin{document}

\pagestyle{fancy}
\fancyhead[RO]{Smirnov M.G.}
\fancyhead[LO]{Group N3264}

\section{Homework №1}

\textbf{The famous person.}

Daniel Kahneman is  Israeli-born psychologist, corecipient of the Nobel Prize for Economics in 2002 for his integration of psychological research into economic science. His pioneering work examined human judgment and decision making under uncertainty. Kahneman shared the award with American economist Vernon L. Smith.

Kahneman began his prizewinning research in the late 1960s. In order to increase understanding of how people make economic decisions, he drew on cognitive psychology in relation to the mental processes used in forming judgments and making choices. Kahneman’s research with Amos Tversky on decision making under uncertainty resulted in the formulation of a new branch of economics, prospect theory, which was the subject of their seminal article “Prospect Theory: An Analysis of Decisions Under Risk”. 

Previously, economists had believed that people’s decisions are determined by the expected gains from each possible future scenario multiplied by its probability of occurring, but if people make an irrational judgment by giving more weight to some scenarios than to others, their decision will be different from that predicted by traditional economic theory. Kahneman’s research (based on surveys and experiments) showed that his subjects were incapable of analyzing complex decision situations when the future consequences were uncertain. Instead, they relied on heuristic shortcuts, or rule of thumb, with few people evaluating their underlying probability.

\section{Homework №2}

\textbf{Exercise 3.}
As the complexity of the systems and the networks are increasing, vulnerabilities are also increasing and the task of securing the networks is becoming more complex. We are dependent on computers today for controlling large money transfers between banks, insurance, markets, telecommunication, electrical power distribution, health and medical fields, nuclear power plants, space research and satellites. We cannot negotiate security in these critical areas. Data is the most precious factor of today’s businesses. Top business organizations spend billions of dollers every year to secure their computer networks and to keep their business data safe. 


\section{Homework №3}

\textbf{Exercise 3.}

Vulnerabilities in network security can be summed up as the “soft spots” that are present in every network. The vulnerabilities are present in the network and individual devices that make up the network. Threats can be internal or external. External threats can arise from individuals or organizations working outside of a company. They do not have authorized access to the computer systems or network. Internal threats occur when someone has authorized access to the network with either an account on a server or physical access to the network.

\section{Homework №4}

\textbf{Exercise 3.}

There are four primary classes of attacks: reconnaissance, access, denial of service and viruses.
Reconnaissance is a type of attack in which an intruder engages with your system to gather information about vulnerabilities. A denial of service is an attack meant to shut down a machine or network, making it inaccessible to its intended users. Malicious software is inserted onto a host to damage a system.

\end{document}
