\section{Заряд. Закон Кулона. Поле силы Кулона.}

\textbf{Электрический заряд} – это физическая величина, характеризующая свойство частиц или тел вступать в электромагнитные силовые взаимодействия.

\textbf{Закон Кулона} - силы взаимодействия неподвижных зарядов прямо пропорциональны произведению модулей зарядов и обратно пропорциональны квадрату расстояния между ними.
\[
\vec{F} =k \cdot \frac{|q_1| \cdot |q_2|}{r^2} \cdot \frac{\vec{r}}{r}, k = \frac{1}{4\pi\varepsilon_0}, \varepsilon_0 = 8.85 \cdot 10^{-12}
\]

\textbf{Принцип суперпозиции} - если заряженное тело взаимодействует одновременно с несколькими заряженными телами, то результирующая сила, действующая на данное тело, равна векторной сумме сил, действующих на это тело со стороны всех других заряженных тел.

По современным представлениям, электрические заряды не действуют друг на друга непосредственно. Каждое заряженное тело создает в окружающем пространстве \textbf{электрическое поле}. Это поле оказывает силовое действие на другие заряженные тела. Главное свойство \textbf{электрического поля} – действие на электрические заряды с некоторой силой. Таким образом, взаимодействие заряженных тел осуществляется не непосредственным их воздействием друг на друга, а через электрические поля, окружающие заряженные тела.

Для количественного определения электрического поля вводится силовая характеристика \textbf{напряженность электрического поля}.

\[
\vec{E}=\frac{\vec{F}}{q}
\]

В соответствии с законом Кулона напряженность электростатического поля, создаваемого точечным зарядом Q на расстоянии r от него, равна по модулю

\[
\vec{E}= \frac{1}{4 \pi \varepsilon_0}\cdot\frac{Q}{r^2}
\]

Это поле называется \textbf{кулоновским}.
