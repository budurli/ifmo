\documentclass[a4paper,12pt]{article}

\usepackage[T2A]{fontenc}
\usepackage[utf8]{inputenc}
\usepackage[english,russian]{babel}
\usepackage{indentfirst}
\usepackage{graphicx}
\usepackage{wrapfig}
\usepackage{amsmath}
\usepackage{lscape}
\usepackage{fancyhdr}
\usepackage{tabularx}

\usepackage{caption}
\begin{document}
	\newpage
	\pagestyle{fancy}
	\fancyhead[RO]{Смирнов М.Г.}
	\fancyhead[LO]{Группа N3264}
	
	\begin{center}
		{\Large Лист измерений}\\
		<<Изучение электрических сигналов с помощью электронного осциллографа>>
	\end{center}

\begin{table}[h]
	\caption{}
	\begin{tabularx}{\textwidth}{|X|X|X|}
		\hline 
		Канал & I & II \\ 
		\hline 
		Цена деления Y-шкалы осциллографа, B/дел &  &  \\ 
		\hline 
		Амплитуда сигнала, измеренная с помощью осциллографа, дел &  &  \\ 
		\hline 
		Амплитуда сигнала, измеренная с помощью осциллографа, U, В &  &  \\ 
		\hline 
		Погрешность измерения амплитуды с помощью осциллографа, $\Delta$U, В &  &  \\ 
		\hline 
		Амплитуда сигнала, измеренная с помощью вольтметра,  U, В &  &  \\ 
		\hline 
		Погрешность измерения амплитуды с помощью вольтметра, $\Delta$U, В &  &  \\ 
		\hline 
		Относительное отклонение показаний осциллографа от показаний вольтметра, \% &  &  \\ 
		\hline 
	\end{tabularx} 
\end{table}

\begin{table}[h]
	\caption{}
	\begin{tabularx}{\textwidth}{|X|X|}
		\hline 
		Канал & II \\ 
		\hline 
		Цена деления X-шкалы осциллографа, mc/дел &  \\ 
		\hline 
		T сигнала ГН1, mc &  \\ 
		\hline 
		$\Delta$T сигнала ГН1, mc &  \\ 
		\hline 
		f сигнала ГН1, Гц &  \\ 
		\hline 
		$\Delta$f сигнала ГН1, Гц &  \\ 
		\hline 
	\end{tabularx} 
\end{table}

\begin{table}[h]
	\caption{}
	\begin{tabularx}{\textwidth}{|X|X|X|X|X|}
		\hline
		\multicolumn{2}{|c|}{Задание} & 		\multicolumn{2}{|c|}{Измеряемые величины} & Вычислить \\ \hline
		Диапазон & Частота грубо & Частота точно по показаниям индикатора (2) ЗГ1 $f \pm \Delta f$, Гц & Картина фигуры Лиcсажу & Частота ГН1 и её погрешность $f \pm \Delta f$, Гц \\ \hline
		50 - 130 Гц & 100 Гц & & & \\ \hline
		130 - 350 Гц & 150 Гц & & & \\ \hline
		130 - 350 Гц & 200 Гц & & & \\ \hline
		350 - 900 Гц & 600 Гц & & & \\ \hline
		350 - 900 Гц & 800 Гц & & & \\ \hline
		900 - 2000 Гц & 1200 Гц & & & \\ \hline
		& & & & $\overline{f} \pm \Delta \overline{f}$, Гц \\ \hline 
	\end{tabularx}
\end{table}

\begin{table}[h]
	\caption{}
	\begin{tabularx}{\textwidth}{|X|X|}
		\hline 
		Канал & II \\ 
		\hline 
		Цена деления X-шкалы осциллографа, mc/дел &  \\ 
		\hline 
		T сигнала ГН1, mc &  \\ 
		\hline 
		$\Delta$T сигнала ГН1, mc &  \\ 
		\hline 
		f сигнала ГН1, Гц &  \\ 
		\hline 
		$\Delta$f сигнала ГН1, Гц &  \\ 
		\hline 
	\end{tabularx} 
\end{table}

\begin{table}[h]
	\caption{}
	\begin{tabularx}{\textwidth}{|X|X|X|X|X|}
		\hline
		\multicolumn{2}{|c|}{Задание} & 		\multicolumn{2}{|c|}{Измеряемые величины} & Вычислить \\ \hline
		Диапазон & Частота грубо & Частота точно по показаниям индикатора (2) ЗГ1 $f \pm \Delta f$, Гц & Картина фигуры Лиcсажу & Частота ГН1 и её погрешность $f \pm \Delta f$, Гц \\ \hline
		50 - 130 Гц & 100 Гц & & & \\ \hline
		130 - 350 Гц & 150 Гц & & & \\ \hline
		130 - 350 Гц & 200 Гц & & & \\ \hline
		350 - 900 Гц & 600 Гц & & & \\ \hline
		350 - 900 Гц & 800 Гц & & & \\ \hline
		900 - 2000 Гц & 1200 Гц & & & \\ \hline
		900 - 2000 Гц & 1800 Гц & & & \\ \hline
		2000 - 5000 Гц & 2400 Гц & & & \\ \hline		
		& & & & $\overline{f} \pm \Delta \overline{f}$, Гц \\ \hline 
	\end{tabularx}
\end{table}

\begin{table}[h]
	\caption{}
	\begin{tabularx}{\textwidth}{|X|X|}
		\hline 
		Канал & II \\ 
		\hline 
		Цена деления X-шкалы осциллографа, mc/дел &  \\ 
		\hline 
		T сигнала ГН1, mc &  \\ 
		\hline 
		$\Delta$T сигнала ГН1, mc &  \\ 
		\hline 
		f сигнала ГН1, Гц &  \\ 
		\hline 
		$\Delta$f сигнала ГН1, Гц &  \\ 
		\hline 
	\end{tabularx} 
\end{table}


\begin{table}[h]
	\caption{}
	\begin{tabularx}{\textwidth}{|X|X|X|}
		\hline
		\multicolumn{2}{|c|}{Измеряемые величины} & Вычислить \\ \hline
		Частота точно по показаниям индикатора (2) ЗГ1 $f \pm \Delta f$, Гц & Картина фигуры Лиссажу & Частота ГН1 и её погрешность $f \pm \Delta f$, Гц \\ \hline
		& & \\ \hline
		& & \\ \hline
		& & \\ \hline
		& & \\ \hline
		& & $\overline{f} \pm \Delta \overline{f}$, Гц  \\ \hline				
	\end{tabularx}
\end{table}


\begin{table}
	\caption{}
	\begin{tabularx}{\textwidth}{|X|X|X|X|X|X|X|}
		\hline
		№ опыта & Цена деления шкалы осц. mc/дел & $T \pm \Delta T$, дел & $T \pm \Delta T$, мс & $\tau \pm \Delta \tau$, дел & $\tau \pm \Delta \tau$, мс & $S \pm \Delta S$ \\ \hline
		1 & & & & & & \\ \hline
		2 & & & & & & \\ \hline
		3 & & & & & & \\ \hline				
	\end{tabularx}
\end{table}


\end{document}