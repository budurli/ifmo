\documentclass[14pt,a4paper,report]{ncc}
\usepackage[a4paper, mag=1000, left=2.5cm, right=1cm, top=2cm, bottom=2cm, headsep=0.7cm, footskip=1cm]{geometry}
\usepackage[utf8]{inputenc}
\usepackage[english,russian]{babel}
\usepackage{indentfirst}
\usepackage[dvipsnames]{xcolor}
\usepackage[colorlinks]{hyperref}
\usepackage{listings} 
\usepackage{caption}
\usepackage{tabularx}
\usepackage{tikz}
\usepackage{pgfplots}
\usepackage{siunitx}
\DeclareCaptionFont{white}{\color{white}} 
\DeclareCaptionFormat{listing}{\colorbox{gray}{\parbox{\textwidth}{#1#2#3}}}
\lstset{% Собственно настройки вида листинга
inputencoding=utf8, extendedchars=\true, keepspaces = true, % поддержка кириллицы и пробелов в комментариях
language=Pascal,            % выбор языка для подсветки (здесь это Pascal)
basicstyle=\small\sffamily, % размер и начертание шрифта для подсветки кода
numbers=left,               % где поставить нумерацию строк (слева\справа)
numberstyle=\tiny,          % размер шрифта для номеров строк
stepnumber=1,               % размер шага между двумя номерами строк
numbersep=5pt,              % как далеко отстоят номера строк от подсвечиваемого кода
backgroundcolor=\color{white}, % цвет фона подсветки - используем \usepackage{color}
showspaces=false,           % показывать или нет пробелы специальными отступами
showstringspaces=false,     % показывать или нет пробелы в строках
showtabs=false,             % показывать или нет табуляцию в строках
frame=single,               % рисовать рамку вокруг кода
tabsize=2,                  % размер табуляции по умолчанию равен 2 пробелам
captionpos=t,               % позиция заголовка вверху [t] или внизу [b] 
breaklines=true,            % автоматически переносить строки (да\нет)
breakatwhitespace=false,    % переносить строки только если есть пробел
escapeinside={\%*}{*)}      % если нужно добавить комментарии в коде
}


\begin{document}
% Переоформление некоторых стандартных названий
\renewcommand{\chaptername}{Лабораторная работа}
\def\contentsname{Содержание}

\begin{titlepage}
	\begin{center}
		\large
			
		Санкт-Петербургский национальный исследовательский университет информационных технологий, механики и оптики
		\vspace{0.25cm}
		
		
		\vfill
		\textsc{Эссе}\\[5mm]
		
		<<Бойцовский клуб>>

		
	\end{center}
	\vfill
	
	\begin{flushright}
Автор:\\
Смирнов М.Г.\\
Факультет БИТ\\
Группа N3264\\
Преподаватель:\\
Тимофеева И.В.

	\end{flushright}
	\vfill
	\begin{center}
		Санкт-Петербург, 2017 г.
	\end{center}
\end{titlepage}


% Содержание
\tableofcontents
\newpage

\chapter{Определение длины световой волны по картине дифракции на круглом отверстии}

\section{Цель работы}

Определение длины световой волны по картине дифракции на малом круглом отверстии

\section{Ход работы}

Координата объектива $X_\infty = 83.4$.

\begin{table}[H]
	\centering
	\caption{Замеры}
	\begin{tabular}{|c|c|c|c|}
		\hline
		$m$ & $X_{o1}$ & $X_{o2}$ & $X_{o3}$ \\ \hline
		2 & 64 & 64.7 & 64.5 \\ \hline
		3 & 70.5 & 70.1 & 70.7 \\ \hline
		4 & 73.6 & 73.4 & 73.5 \\ \hline
		5 & 75.7 & 75.6 & 75.7 \\ \hline
		6 & 77 & 77 & 77.1 \\ \hline
	\end{tabular}
\end{table}

\[
d=l-b=x_\infty-x
\]
\begin{table}[H]
	\centering
	\caption{Расчёты}
	\begin{tabular}{|c|c|c|c|}
		\hline
		$m$ & $d_1$ & $d_2$ & $d_3$ \\ \hline
		2 & 19.4 & 18.7 & 18.9 \\ \hline
		3 & 12.9 & 13.3 & 12.7 \\ \hline
		4 & 9.8 & 10 & 9.9 \\ \hline
		5 & 7.7 & 7.8 & 7.7 \\ \hline
		6 & 6.4 & 6.4 & 6.3 \\ \hline
	\end{tabular}
\end{table}

\begin{table}[H]
	\centering
	\caption{Зависимость}
	\begin{tabular}{|c|c|c|c|c|c|}
		\hline
		$\frac{1}{m}$ & $\frac{1}{6}$ & $\frac{1}{5}$ & $\frac{1}{4}$ & $\frac{1}{3}$ & $\frac{1}{2}$ \\ \hline
		$d$ & 6.37 & 7.73 & 9.9 & 12.97 & 19 \\ \hline
	\end{tabular}
\end{table}

\begin{landscape}
	\begin{tikzpicture}
	\begin{axis}[grid=both,
	width=0.8\linewidth]
	\addplot coordinates {
		(1/6, 6.37)
		(1/5, 7.73)
		(1/4, 9.9)
		(1/3, 12.97)
		(1/2, 19)		
	};
	\end{axis}
	
	\end{tikzpicture}
\end{landscape}

Уравнение аппроксимирующей прямой:

\[
y=37.7309 \cdot x + 0.252045
\] 

Длина световой волны:

\[
\lambda = \frac{r^2}{k} = \frac{(0.5*10^{-1})^2}{37.7309} = 662 \text{нм}
\]

Погрешность наклона $\Delta K$:

\[
\Delta K = \sqrt{(k-k_1)^2+(k-k_2)^2+(k-k_3)^2+(k-k_4)^2} = 0.0182
\]

\[
\Delta \lambda = \sqrt{\frac{0.0005^2}{37.739^2}+\frac{2*0.0005}{37.739}} \approx 53.3 \text{ нм}
\]


Длина волны с учётом погрешности:

\[
\lambda = 662 \pm 53.3 \text{ нм}
\]




\end{document}