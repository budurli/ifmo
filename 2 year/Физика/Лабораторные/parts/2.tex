\chapter{Изучение электрических сигналов с помощью электронного осциллографа}

\section{Цель работы}

Ознакомление с устройством осциллографа, изучение с его помощью процессов в электрических цепях.

\section{Ход работы}

\begin{table}[h]
	\caption{}
	\begin{tabular}{|p{10cm}|c|c|}
		\hline
		Канал & I & II \\ \hline
		Масштаб, В/дел & 1 & 0.5 \\ \hline
		Амплитуда сигнала, дел & 1.8 & 1.7 \\ \hline
		Амплитуда $U_m$, В & 1.8 & 1.1 \\ \hline
		Погрешность $\Delta U$, В & $\pm 1$ & $\pm 0.1$ \\ \hline
		Амплитуда $U_{mv}$, В & 1.8 & 1.04 \\ \hline
		Погрешность $\Delta U_{mv}$, В & $\pm 0.01 $ & $\pm 0.01$ \\ \hline
		Относительное отклонение $(U_m-U_{mv})/U_{mv}$ & 0\% & 0.5\% \\ \hline
	\end{tabular}
\end{table}

\begin{table}[h]
	\caption{}
	\begin{tabular}{|p{10cm}|c|c|}
		\hline
		Канал & I & II \\ \hline
		Масштаб X, мс/дел & 0.5 & 0.5 \\ \hline
		T, дел & 2.6 & 2.7 \\ \hline
		T, мс & 1.3 & 1.35 \\ \hline
		$\Delta T$ & $\pm 0.05$ & $\pm 0.05$ \\ \hline
		$f$, Гц & 792 & - \\ \hline
		$\Delta f$, Гц & $\pm 0.02$ & - \\ \hline
		$T_\text{г} = 1/f$ & 0.0013 & - \\ \hline
		$\Delta T_\text{г}$, мс & 0.0014 & - \\ \hline
		$\Delta T_\text{г}/T_\text{г}$ & 0.1 & - \\ \hline
		$(T-T_\text{г})/T_\text{г}$ & 5\% & - \\ \hline
	\end{tabular}
\end{table}

\begin{table}[h]
	\caption{}
	\begin{tabular}{|p{4cm}|p{4cm}|p{4cm}|p{4cm}|}
		\hline
		Диапазон частот, Гц & Частота сигнала грубо, Гц & Вид фигуры & Частота сигнала точно, Гц \\ \hline
		$50 \dots 130  $ & 100 & & 127.5 \\ [50pt] \hline
		$130 \dots 350  $ & 133 & & 128 \\ [50pt] \hline
		$350 \dots 900  $ & 600 & & 603.5 \\ [50pt] \hline
		$130 \dots 350  $ & 200 & & 209 \\ [50pt] \hline
		$350 \dots 900  $ & 800 & & 821 \\ [50pt] \hline
		$900 \dots 2000  $ & 1200 & & 1249 \\ [50pt] \hline		
	\end{tabular}
\end{table}