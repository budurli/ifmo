\documentclass[14pt,a4paper,report]{ncc}
\usepackage[a4paper, mag=1000, left=2.5cm, right=1cm, top=2cm, bottom=2cm, headsep=0.7cm, footskip=1cm]{geometry}
\usepackage[utf8]{inputenc}
\usepackage[english,russian]{babel}
\usepackage{indentfirst}
\usepackage[dvipsnames]{xcolor}
\usepackage[colorlinks]{hyperref}
\usepackage{listings} 
\usepackage{caption}
\usepackage{tabularx}
\usepackage{tikz}
\usepackage{pgfplots}
\usepackage{siunitx}
\usepackage{amsmath}
\DeclareCaptionFont{white}{\color{white}} 
\DeclareCaptionFormat{listing}{\colorbox{gray}{\parbox{\textwidth}{#1#2#3}}}
\lstset{% Собственно настройки вида листинга
inputencoding=utf8, extendedchars=\true, keepspaces = true, % поддержка кириллицы и пробелов в комментариях
language=Pascal,            % выбор языка для подсветки (здесь это Pascal)
basicstyle=\small\sffamily, % размер и начертание шрифта для подсветки кода
numbers=left,               % где поставить нумерацию строк (слева\справа)
numberstyle=\tiny,          % размер шрифта для номеров строк
stepnumber=1,               % размер шага между двумя номерами строк
numbersep=5pt,              % как далеко отстоят номера строк от подсвечиваемого кода
backgroundcolor=\color{white}, % цвет фона подсветки - используем \usepackage{color}
showspaces=false,           % показывать или нет пробелы специальными отступами
showstringspaces=false,     % показывать или нет пробелы в строках
showtabs=false,             % показывать или нет табуляцию в строках
frame=single,               % рисовать рамку вокруг кода
tabsize=2,                  % размер табуляции по умолчанию равен 2 пробелам
captionpos=t,               % позиция заголовка вверху [t] или внизу [b] 
breaklines=true,            % автоматически переносить строки (да\нет)
breakatwhitespace=false,    % переносить строки только если есть пробел
escapeinside={\%*}{*)}      % если нужно добавить комментарии в коде
}


\begin{document}
% Переоформление некоторых стандартных названий
\renewcommand{\chaptername}{Лабораторная работа}
\def\contentsname{Содержание}

\begin{titlepage}
	\begin{center}
		\large
			
		Санкт-Петербургский национальный исследовательский университет информационных технологий, механики и оптики
		\vspace{0.25cm}
		
		
		\vfill
		\textsc{Эссе}\\[5mm]
		
		<<Бойцовский клуб>>

		
	\end{center}
	\vfill
	
	\begin{flushright}
Автор:\\
Смирнов М.Г.\\
Факультет БИТ\\
Группа N3264\\
Преподаватель:\\
Тимофеева И.В.

	\end{flushright}
	\vfill
	\begin{center}
		Санкт-Петербург, 2017 г.
	\end{center}
\end{titlepage}


% Содержание
\tableofcontents
\newpage

\chapter{Первая}
\section{Цель работы}
\section{Ход работы}
\chapter{Изучение электрических сигналов с помощью электронного осциллографа}

\section{Цель работы}

Ознакомление с устройством осциллографа, изучение с его помощью процессов в электрических цепях.

\section{Ход работы}

\begin{table}[h]
	\caption{}
	\begin{tabular}{|p{10cm}|c|c|}
		\hline
		Канал & I & II \\ \hline
		Масштаб, В/дел & 1 & 0.5 \\ \hline
		Амплитуда сигнала, дел & 1.8 & 1.7 \\ \hline
		Амплитуда $U_m$, В & 1.8 & 1.1 \\ \hline
		Погрешность $\Delta U$, В & $\pm 1$ & $\pm 0.1$ \\ \hline
		Амплитуда $U_{mv}$, В & 1.8 & 1.04 \\ \hline
		Погрешность $\Delta U_{mv}$, В & $\pm 0.01 $ & $\pm 0.01$ \\ \hline
		Относительное отклонение $(U_m-U_{mv})/U_{mv}$ & 0\% & 0.5\% \\ \hline
	\end{tabular}
\end{table}

\begin{table}[h]
	\caption{}
	\begin{tabular}{|p{10cm}|c|c|}
		\hline
		Канал & I & II \\ \hline
		Масштаб X, мс/дел & 0.5 & 0.5 \\ \hline
		T, дел & 2.6 & 2.7 \\ \hline
		T, мс & 1.3 & 1.35 \\ \hline
		$\Delta T$ & $\pm 0.05$ & $\pm 0.05$ \\ \hline
		$f$, Гц & 792 & - \\ \hline
		$\Delta f$, Гц & $\pm 0.02$ & - \\ \hline
		$T_\text{г} = 1/f$ & 0.0013 & - \\ \hline
		$\Delta T_\text{г}$, мс & 0.0014 & - \\ \hline
		$\Delta T_\text{г}/T_\text{г}$ & 0.1 & - \\ \hline
		$(T-T_\text{г})/T_\text{г}$ & 5\% & - \\ \hline
	\end{tabular}
\end{table}

\begin{table}[h]
	\caption{}
	\begin{tabular}{|p{4cm}|p{4cm}|p{4cm}|p{4cm}|}
		\hline
		Диапазон частот, Гц & Частота сигнала грубо, Гц & Вид фигуры & Частота сигнала точно, Гц \\ \hline
		$50 \dots 130  $ & 100 & & 127.5 \\ [50pt] \hline
		$130 \dots 350  $ & 133 & & 128 \\ [50pt] \hline
		$350 \dots 900  $ & 600 & & 603.5 \\ [50pt] \hline
		$130 \dots 350  $ & 200 & & 209 \\ [50pt] \hline
		$350 \dots 900  $ & 800 & & 821 \\ [50pt] \hline
		$900 \dots 2000  $ & 1200 & & 1249 \\ [50pt] \hline		
	\end{tabular}
\end{table}
\chapter{Исследования характеристик источника тока}

\section{Цель работы}
Исследовать зависимость полной мощности, полезной мощности, мощности потерь, падения напряжения во внешней цепи и КПД источника от силы тока в цепи

\section{Ход работы}

\begin{table}[h]
	\caption{}
	\centering
	\begin{tabular}{|c|c|c|c|c|c|c|c|c|c|c|c|c|c|c|c|}
		\hline
		R, Ом & 100 & 200 & 300 & 400 & 500 & 600 & 700 & 800 & 900 & 1000 \\ \hline
		U, В & 19.2 & 16.9 & 15.0 & 13.2 & 12.0 & 11.1 & 10.3 & 9.5 & 8.8 & 8.1 \\ \hline
		I, А & 1.007 & 2.23 & 3.54 & 4.76 & 5.53 & 6.14 & 6.69 & 7.20 & 7.69 & 8.13    \\ \hline
	\end{tabular}
\end{table}

\begin{figure}[h]
	\centering
	\caption{График зависимости напряжения $U$ от силы тока $I$}
	\begin{tikzpicture}
	\begin{axis}[
	xlabel=$I$,
	ylabel=$U$,
	xmin=0, xmax=15,
	ymin=0, ymax=25,
	grid=major
	]
	\addplot [smooth, thick, domain=0:15]{20.5-1.52*x};
	\addplot coordinates {
	(1.00, 19.2)
	(2.23, 16.9)
	(3.54, 15.0)
	(4.76, 13.2)
	(5.53, 12.0)
	(6.14, 11.1)
	(6.69, 10.3)
	(7.20, 9.5)
	(7.69, 8.8)
	(8.13, 8.1)	
	};
	\end{axis}
	\end{tikzpicture}
\end{figure}

Экстраполируя график до пересечения с осями координат, получаем:
\[
\varepsilon = 20.5 \text{ [В]}, I_K = 13.4 \text{ [А]}
\]

Определим внутреннее сопротивление источника $r$
\[
r = \frac{\varepsilon}{I_K} = \frac{20.5}{13.4} = 1.529
\]

Рассчитаем мощности $P$, $P_1$, $P_2$ и КПД $\eta$ для измеренных значений силы тока. Построим график зависимостей этих величин от силы тока.

\[
P = P_1+ P_2, P_1 = \varepsilon \cdot I  - I^2\cdot r, P_2 = I^2 \cdot r
\]

\begin{figure}[h]
	\centering
	\caption{График зависимости мощностей от силы тока $I$}
	\begin{tikzpicture}
	\begin{axis}[
	xlabel=$I$,
	ylabel=$P$,
	xmin=0, xmax=15,
	ymin=0, ymax=200,
	grid=major
	]
	\addplot [domain=0:14]{20.5 * x};
	\addplot [domain=0:14]{20.5 * x- x * x * 1.529};
	\addplot [domain=0:14]{x * x * 1.529};	
	\end{axis}
	\end{tikzpicture}
\end{figure}

\begin{figure}[h]
	\centering
	\caption{Зависимость КПД источника от силы тока в замкнутой цепи}
	\begin{tikzpicture}
	\begin{axis}[
	xmin=0, xmax=15,
	ymin=0,
	grid=major,
	xlabel=$I$,
	ylabel=$\eta$]
	\addplot [domain=0:20]{1-x*1.49/20};
	\end{axis}
	\end{tikzpicture}
\end{figure}
\chapter{Изучение картины эквипотенциальных поверхностей и силовых	линий электростатического поля с помощью электролитической ванны}

\section{Цель работы}

Осуществить построение эквипотенциальных линий электростатического поля с помощью экспериментального моделирования в проводящей среде, в которой протекает переменный ток

\section{Ход работы}
\chapter{Изучение свободных затухающих электромагнитных колебаний}

\section{Цель работы}

Изучение основных характеристик свободных затухающих колебаний.

\section{Ход работы}

\[
\lambda=\frac{1}{n}\cdot \ln\frac{U_i}{U_{i+n}}, Q = \frac{2\cdot \pi}{1-e^{-2 \lambda}}, R=R_m + R_0, L = \frac{\pi^2 \cdot R^2 \cdot C}{\lambda ^ 2}
\]

\begin{table}[h]
	\caption{}
	\begin{tabularx}{\textwidth}{|X|X|X|X|X|X|X|X|X|}
		\hline 
		$R_\text{м}$, Ом & T, мс & $2U_i$, дел & $2U_{i+n}$, дел & n & $\lambda$ & Q & R, Ом & L, мГн \\ 
		\hline 
		0 & 95  & 4.9  & 3.1  & 2 & 0.229 & 17.090 & 50 & 10.340  \\ 
		\hline 
		10 & 95  & 4.6 & 2.9 & 2 & 0.231 & 16.974 & 60 & 14.633  \\ 
		\hline 
		20 & 95 & 4.5  & 2.6  & 2  & 0.274 & 14.885 &  70 & 14.157 \\ 
		\hline 
		30 & 95  & 4.3 & 2.2 & 2 & 0.335 & 12.861 & 80 & 12.370 \\ 
		\hline 
		40 & 95 & 4.2 & 2.1 & 2  & 0.347  & 12.543 & 90 & 14.591  \\ 
		\hline 
		50 & 95 & 4.0 & 1.9 & 2 & 0.372  & 11.976 & 100 & 15.674 \\ 
		\hline 
		60 & 95 & 3.9 & 1.6 & 2 & 0.445 & 10.655 & 110  & 13.254  \\ 
		\hline 
		70 & 95 & 3.8 & 1.4 & 2 & 0.499 & 9.946 & 120  & 12.544  \\ 
		\hline 
		80 & 95 & 3.6 & 1.3 & 2 & 0.509 & 9.832 & 130  & 14.149  \\ 
		\hline 
		90 & 95 & 3.5 & 1.2 & 2 & 0.535 & 9.558 & 140 & 14.853 \\ 
		\hline 
		100 & 95 & 3.4 & 1 & 2 & 0.612 & 8.895 & 150 & 13.030  \\ 
		\hline 
		200 & 95 & 2.5 & 0.5 & 2  & 0.805 & 7.849 & 250  & 20.920  \\ 
		\hline 
		300 & 95 & 1.7 & 0.2 & 1 & 2.140 & 6.368 & 350 & 5.802  \\ 
		\hline 
		400 & 95 & 1.3 & 0.1 & 1 & 2.565 & 6.317 & 450 & 6.676 \\ 
		\hline 
	\end{tabularx} 
\end{table}

\begin{table}[h]
	\caption{}
	\begin{tabularx}{\textwidth}{|X|X|X|X|}
		\hline 
		С, мкФ & $T_\text{эксп}$, мс & $T_\text{теор}$, мс & $\delta T=\frac{T_\text{эксп}-T_\text{теор}}{T_\text{теор}}$, \% \\ 
		\hline 
		0.022 & 0.9 & 0.8179 & 0.1  \\ 
		\hline 
		0.033 & 1.1 & 1.0653  & 0.03  \\ 
		\hline 
		0.047 & 1.3 & 1.278  & 0.017 \\ 
		\hline 
		0.47 & 4.2 & 3.714 & 0.115  \\ 
		\hline 
	\end{tabularx} 
\end{table}

\begin{figure}[h]
	\centering
	\caption{График зависимости $\lambda$ от $R_m$}
	\begin{tikzpicture}
	\begin{axis}[
	xlabel=$R$,
	ylabel=$\lambda$,
	xmin=-60, xmax= 120,
	ymin=0, ymax=1,
	grid=major
	]
	\addplot[domain=-50:110, samples=100] {0.004*x + 0.2};
	\addplot coordinates {
		(0, 0.229)
		(10, 0.231)
		(20, 0.274)
		(30, 0.335)
		(40, 0.347)
		(50, 0.372)
		(60, 0.445)
		(70, 0.499)
		(80, 0.509)
		(90, 0.535)
		(100, 0.612)
	};
	\end{axis}
	\end{tikzpicture}
\end{figure}

\begin{figure}[h]
	\centering
	\caption{График зависимости Q от R}
	\begin{tikzpicture}
	\begin{axis}[
	xlabel=$R$,
	ylabel=$Q$,
	xmin=0, xmax=500,
	ymin=0, ymax=20,
	grid=major
	]
	\addplot coordinates {
		(50, 17.09064326)
		(60, 16.97399783)
		(70, 14.88521548)
		(80, 12.86117207)
		(90, 12.54930227)
		(100, 11.96667483)
		(110, 10.65591126)
		(120, 9.946402453)
		(130, 9.83273164)
		(140, 9.558723616)
		(150, 8.895834376)
		(250, 7.848897368)
		(350, 6.368152179)
		(450, 6.317377144)
	};
	\end{axis}
	\end{tikzpicture}
\end{figure}

\end{document}