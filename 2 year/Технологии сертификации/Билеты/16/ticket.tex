\section{Порядок проведения сертификации СЗИ.}

В общем виде процедура сертификации включает:
\begin{itemize}
\item подачу и рассмотрение заявки на проведение сертификации (продление срока
действия сертификата) средств защиты информации;
\item сертификационные испытания средств защиты информации и (при необходимости)
аттестацию их производства;
\item экспертизу результатов испытаний, оформление, регистрацию и выдачу сертификата и
лицензии на право использования знака соответствия;
\item осуществление государственного контроля и надзора, инспекционного контроля за
соблюдением правил обязательной сертификации и за сертифицированными
средствами защиты информации;
\item информирование о результатах сертификации средств защиты информации;
\item рассмотрение апелляций.
\end{itemize}

Этапы:

\begin{itemize}
	\item Заявка на сертификацию
	\item Решение на сертификацию
	\item Договор на проведение сертификационных испытаний
	\item Подготовка исходных данных
	\item Сертификационные испытания
	\item Оформление результатов сертификационных испытаний
	\item Договор на проведение экспертизы результатов испытаний
	\item Экспертиза результатов испытаний
	\item Оформление сертификата
\end{itemize}