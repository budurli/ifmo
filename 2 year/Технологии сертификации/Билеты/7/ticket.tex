\section{Требования к средствам антивирусной защиты.}

Требования к средствам антивирусной защиты включают общие требования к средствам антивирусной защиты и требования к функциям безопасности средств антивирусной защиты.

Для дифференциации требований к функциям безопасности средств антивирусной защиты установлено шесть классов защиты средств антивирусной защиты. Самый низкий класс – шестой, самый высокий – первый.

Средства антивирусной защиты, соответствующие 6 классу защиты, применяются в информационных системах персональных данных 3 и 4 классов*.

Средства антивирусной защиты, соответствующие 5 классу защиты, применяются в информационных системах персональных данных 2 класса*.

Средства антивирусной защиты, соответствующие 4 классу защиты, применяются в государственных информационных системах, в которых обрабатывается информация ограниченного доступа, не содержащая сведения, составляющие государственную тайну, в информационных системах персональных данных 1 класса*, а также в информационных системах общего пользования II класса**.

Средства антивирусной защиты, соответствующие 3, 2 и 1 классам защиты, применяются в информационных системах, в которых обрабатывается информация, содержащая сведения, составляющие государственную тайну.

Также выделяются следующие типы средств антивирусной защиты:

тип «А» – средства антивирусной защиты (компоненты средств антивирусной защиты), предназначенные для централизованного администрирования средствами антивирусной защиты, установленными на компонентах информационных систем (серверах, автоматизированных рабочих местах);

тип «Б» – средства антивирусной защиты (компоненты средств антивирусной защиты), предназначенные для применения на серверах информационных систем;

тип «В» – средства антивирусной защиты (компоненты средств антивирусной защиты), предназначенные для применения на автоматизированных рабочих местах информационных систем;

тип «Г» – средства антивирусной защиты (компоненты средств антивирусной защиты), предназначенные для применения на автономных  автоматизированных рабочих местах.

Средства антивирусной защиты типа «А» не применяются в информационных системах самостоятельно и предназначены для использования только совместно со средствами антивирусной защиты типов «Б» и (или) «В».