Выпущенный в 1999 году фильм рассказывает о плывущем по течению <<белом воротничке>> в исполнении Эдварда Нортона в переломный момент его судьбы, когда он обращается к <<ультра-насилию>>, маскулинности и абсолютному нигилизму, пропагандируемому Тайлером Дерденом в исполнении Брэда Питта.

Не так много фильмов вызвали столь же сильно противоположные мнения, как <<Бойцовский клуб>>. Одни называют его как постмодернистский шедевром, который документирует жестокую эскалацию мужчины современной потребительской культурой и способы, которыми он может сопротивляться. Другие считают фильм искусственной псевдо-интеллектуальной и лицемерной попыткой голливудской машины обратиться к мужским <<животным>> импульсам, а в конце концов придти к уроку моралистов.

Однако все сходятся во мнении, что у фильма есть чёткий посыл. Очень неоднозначный, но явственный. Но что это за посыл, и помогает ли его двусмысленность?

Протагонист фильма - шестерёнка в огромном механизме неназываемой корпорации. Он живет в кондоминиуме смебелью IKEA. Он переполнен требованиями внешнего мира покупать больше, потреблять больше, чтобы быть больше. Он мучается бессоницей. Чтобы справиться с этим , он отправляется на ночные встречи различных групп поддержки для людей с серьезными заболеваниями. Некоторое время это, похоже, работает; как он сам отмечает: «Каждый вечер я умирал, и каждый вечер я рождался снова, воскресал». Эти ранние сцены ясно иллюстрируют человека, который не справиося с жизнью в современном общества. Только на контрасте с реальной болью других людей, он может чувствовать что-то для себя. Без боли он мертв, с ней же он чувствует себя живым.

Происходит два важнейших  для повествования события. Первое  - встреча с Марлой Сингер, еще одним <<притворщиком>> на групповых встречах, которая, кстати, становится единственным женским персонажем в фильме. Второе событие - самое важное в первой части - встреча с производителем кустарного мыла Тайлером Дерденом. Пересказывать дальнейший сюжет бессмысленно, суть не в истории, а в отношении к ней персонажей.

В «Бойцовском клубе» вытаскивают на поверхность многие проблемы общества, но главная среди них -  роль мужчин. Бойцовский клуб призван привести людей в их более зверское, естественное состояние - состояние, в котором они должны были быть. Потому что «только после катастрофы мы можем воскреснуть».
