\chapter{Расчёт схемы}

Полоса пропускания, рад/с:
\[
\omega =2\cdot  \pi \cdot f=1256.64 
\]

Примем следующие допущения: $C_1=C_2=C$ и $R_1=R_2=R$.

Выбран усилитель с низким показателем уровня шума, с удовлетворяющей требованиям частотной характеристикой \textbf{OPA27}. Все характеристики представлены в разделе <<Выбор компонентов>>.


Табулированные значения коэффициентов принимают следующие значения:

Передаточная функция схемы имеет вид:

\[
H(p)=\frac{U_{out} (p)}{U_{in} (p)}=\frac{K}{R_1 R_2 C_1 C_2 \omega^2 \rho^2+(C_1 (R_1+R_2 )+(1-K) R_1 C_2)\omega \rho+1}
\]

\[
a_1=1.3617
\]

\[
b_1=0.6180
\]

Примем условие, что $K=1$. В таком случае $R_3$ – перемычка, а $R_4$ не является необходимым. При этом передаточная функция каскада принимает вид:

\[
H(p)=\frac{U_{out} (p)}{U_{in} (p)}=\frac{1}{R_1 R_2 C_1 C_2 \omega^2 \rho^2+(C_1 (R_1+R_2 )+(1-K) R_1 C_2)\omega \rho+1}
\]

Чтобы значения $R_1$ и $R_2$ были действительными, должно выполняться условие:
\[
\frac{C_2}{C_1} \geq \frac{4\cdot b_1}{a_1^2} = \frac{4}{3}
\]

Выбраны конденсаторы \textbf{К10-17Б} ёмкостью 2,2 мкФ и К10-17Б ёмкостью 1 мкФ Все параметры представлены в разделе <<Выбор компонентов>>.

Резисторы выбираются по условию:
\[
R_1=R_2=\frac{a_1 \cdot C_2 \pm \sqrt{(a_1\cdot C_2)^2-4\cdot b_1 \cdot C_1 \cdot C_2}}{2 \omega \cdot C_1 c\dot C_2} \approx 200
\]
