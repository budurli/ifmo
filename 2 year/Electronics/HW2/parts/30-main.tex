\chapter{Расчёт схемы}
Максимальное и минимальное входное напряжение:

\[
U_{IN\_MAX} = 22 \text{ В}
\]
\[
U_{IN\_MIN} = 20 \text{ В}
\]

Ток нагрузки:

\[
I_L=\frac{U_{OUT}}{R_L} =0.0375 \text{ А}
\]

Максимальное напряжение коллектор-эмиттер:

\[
U_{CE\_MAX}>(1.5…2) U_{IN\_MAX}=[33…44] \text{ В}
\]

Максимальный ток коллектора:
\[
I_{C\_MAX}>2 \cdot I_L=0.075 \text{ A}
\]

Выбран транзистор \textbf{MMBT3904T}, параметры представлены в разделе <<Компоненты>>.

\[
h_{21}=50
\]

Напряжение стабилизации:

\[
U_{ST}=U_{OUT}-U_{BE}=14.3 \text{ В}
\]

Выбран стабилитрон \textbf{BZT52C15} с параметрами:

\[
U_{ST\_MAX}=13.8 \text{ В}
\]

\[
U_{ST\_MIN}=15.6 \text{ В}
\]

\[
I_{ST\_MIN}=0.05 \text{ А}
\]

Прочие параметры представлены в разделе <<Компоненты>>.

Ток через балластный резистор:

\[
I_{R_o}=2 \cdot I_{ST\_MIN} \cdot  (1+h_{21})=0.51 \text{ А}
\]

Сопротивление балластного резистора, Ом:

\[
R_o \leq \frac{U_{IN\_MIN}-U_{OUT}}{I_{R_o}} =10 Ом
\]

Выбран резистор \textbf{CF-100} (С1-4) с ёмкостным номиналом 10 Ом. Параметры представлены в разделе <<Компоненты>>.

Коэффициент стабилизации:

\[
K_{ST}=\frac{U_{OUT}}{U_{IN}} \cdot   \frac{R_O}{r_D} =3.69
\]

Амплитуда выходного напряжения, В:
\[
\Delta U_{OUT}=\frac{\Delta U_{IN} \cdot r_D}{R_o} = 0.19
\]
