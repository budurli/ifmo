\documentclass[a4paper,12pt]{article}

\usepackage[T2A]{fontenc}
\usepackage[utf8]{inputenc}
\usepackage[english,russian]{babel}
\usepackage{indentfirst}
\usepackage{graphicx}
\usepackage{wrapfig}
\usepackage{amsmath}
\usepackage{lscape}
\usepackage{fancyhdr}
\usepackage{tabularx}
\usepackage{caption}
\usepackage{tikz}

\usetikzlibrary{circuits}
\usetikzlibrary{circuits.ee}
\usetikzlibrary{circuits.ee.IEC}
\usetikzlibrary{arrows}
\usetikzlibrary{patterns}




\begin{document}
	\begin{figure}[!h]
		\caption{}
		\begin{center}
			\begin{tikzpicture}[circuit ee IEC]
			\node (R1) at (2,5) [resistor={info = $R_1$}] {};
			\node (R2) at (1,4) [point up, resistor={info = $R_2$}] {};
			\node (R7) at (4,5) [resistor={info = $R_7$}] {};
			\node (L9) at (5,4) [point down, inductor={info = $L_9$}] {};
			\node (E) at (1,2) [point up, voltage source={info = $E$}] {};
			\draw (E) -- (R2) -- (1, 5) -- (R1) -- (R7) -- (5,5) --(L9) -- (5,1) -- (1,1) -- (E) ;
			\draw (3,1) -- (3, 5);
			\end{tikzpicture}
		\end{center}
	\end{figure}
\textit{Дано}: $R_1=R_2=R_7=200$ [Ом], $L_9=0,2$ [Гн], $E=270$ [В].
\textit{Найти}: $i(t), u(t)$ классическим методом расчета.
\end{document}