\section{Атака на алгоритм шифрования RSA посредством метода Ферма}

\textbf{Цель работы}: изучить атаку на алгоритм шифрования RSA посредством метода Ферма.

Дано: $N=\seqsplit{84920690254116819980017474476393118148268821013112888110434002286862060838164293058912544765163219640705110180259944198609255581868934872999810160228209843869386758250683376679520802497046978250461593283557666939606457018130540475876808030500520994415313016543600381848593348673332775443216238563477836798692543716771690964018098579428819953280204930478963564013572076789009718840237654608474337325427262588758054279994682489408386577557217078829277198874401836992816519924056484943107623171715299799789082009713473704559057405215272897426454335415434721217841792235406225913392971814851935883371979117772731283861673}$, $e=11561117$, $C=\seqsplit{27606839229159724202192897590006103812175960309073658988490739528436085374736770037813503925034009698846392122477248596931626015020066651034986067469365190089891542735891189761202525567461005409556622542826876017819147586467351724238454604270020617776122002401970064516181704513310158179809872040150188506260998473827708184953678252898557954311387298338802119127028279839502012359496850898512894890191130340538248388217450552607781712681065657038654205924527107627523433154746012607787256740267693806084449466346764253793666668749716689217513724983160321295430932225088571383388154517247768340767597728787736870516150}$.

Вычисляем $n = [sqrt(N)] + 1$

A = N, B = 2, $D = A^{1/B} = 
\seqsplit{291411547907966097437475668012142221876298569079318524242183274281819560023886760140878408980756408066198439646241201452287857402991893383345901903898208666019418834716389216639770657980067217395575248166874392434687453014189014873271538724919595817010651636073850001657701568380505256583708704660403902559824}$

В первой строке таблицы появляется сообщение <<[error]>>. Это свидетельствует о том, что N не является квадратом целого числа.

$t_1 = n + 1$

Возводим число $t1$ в квадрат:

$A = 
\seqsplit{291411547907966097437475668012142221876298569079318524242183274281819560023886760140878408980756408066198439646241201452287857402991893383345901903898208666019418834716389216639770657980067217395575248166874392434687453014189014873271538724919595817010651636073850001657701568380505256583708704660403902559825}$

$B = 2$

$C = 0 $

$D = A^B \mod C = t_1^2 =
\seqsplit{84920690254116819980017474476393118148268821013112888110434002286862060838164293058912544765163219640705110180259944198609255581868934872999810160228209843869386758250683376679520802497046978250461593283557666939606457018130540475876808030500520994415313016543600381848593348673332775443216238563477836798693491341484316326367696792323609162882891818766325608811365088727920620660427829598196956096469067001914231713847312429543954001417018840412329762687755621977976425467110638565999100198794155067232638630796559241227735122216750286479750488330297235381840884921062606668844521377162476008780949846833187704030625}$

Вычисляем $w_1 = t_1^2 – N$ 

$A = t_1^2$

$B = –N$

$D = A + B = w_1 = 
\seqsplit{947624712625362349598212894789209602686888287362044797793011938910901820190174989722618771041804413156177433852629940135567423859801761583052563813353784985159905543054153622891477027078855267443556621083085536668677717001477389053296152914862514163999092685656380755451549562310540125408970729060456420168952}$

Проверяем, является ли $w_1$ квадратом целого числа: 

$A = w_1$

$B = 2$

$D = A^{1/B} =  «[error]»$

$t_2 = n + 2$
 
$A = 
\seqsplit{291411547907966097437475668012142221876298569079318524242183274281819560023886760140878408980756408066198439646241201452287857402991893383345901903898208666019418834716389216639770657980067217395575248166874392434687453014189014873271538724919595817010651636073850001657701568380505256583708704660403902559826}$

$B = 2$

$C = 0$
 
$D = A^B \mod C = t_2^2 = \seqsplit{84920690254116819980017474476393118148268821013112888110434002286862060838164293058912544765163219640705110180259944198609255581868934872999810160228209843869386758250683376679520802497046978250461593283557666939606457018130540475876808030500520994415313016543600381848593348673332775443216238563477836798694074164580132258562571743659633447326644415904484245859849455276484259780475603118478712914430579818046628593139794832448529716223002627179021566495552039310015263136543416999278641514754289502023789127130308026097110028245128316226293565780136427015862188193210306672159924513923486521948367256153995509150276}$

Вычисляем $w_2 = t_2^2 – N$

$A = t_2^2$

$B = –N$

$D = A + B = w_2 = 
\seqsplit{1530447808441294544473164230813494046439485425520681846277378487474540940237948510004375589003317229288574313145112343040143138665785548349744367621150202317198743212486932056171018343038989702234707117416834321538052623029855418799839230364701705798020395957804080758766952699071550638576388138381264225288603}$

Проверяем, является ли $w_2$ квадратом целого числа: 

$A = w_2$

$B = 2$

$D = A^{1/B} =  «[error]»$

$t_3 = n + 3$
 
$A = \seqsplit{291411547907966097437475668012142221876298569079318524242183274281819560023886760140878408980756408066198439646241201452287857402991893383345901903898208666019418834716389216639770657980067217395575248166874392434687453014189014873271538724919595817010651636073850001657701568380505256583708704660403902559827}$

$B = 2$

$C = 0 $

$D = A^B \mod C = t_3^2 =
\seqsplit{84920690254116819980017474476393118148268821013112888110434002286862060838164293058912544765163219640705110180259944198609255581868934872999810160228209843869386758250683376679520802497046978250461593283557666939606457018130540475876808030500520994415313016543600381848593348673332775443216238563477836798694656987675948190757446694995657731770397013042642882908333821825047898900523376638760469732392092634179025472432277235353105431028986413945713370303348456642054100805976195432558182830714423936814939623464056810966484934273506345972836643229975618649883491465358006675475327650684497035115784665474803314269929}$

Вычисляем $w_3 = t_3^2 – N$

$A = t_3^2$

$B = –N$

$D = A + B = w_3 = \seqsplit{2113270904257226739348115566837778490192082563679318894761745036038180060285722030286132406964830045420971192437594745944718853471769335116436171428946619649237580881919710489450559658999124137025857613750583106407427529058233448546382307814540897432041699229951780762082355835832561151743805547702072030408256}$

Проверяем, является ли $w_3$ квадратом целого числа: 

$A = w_3$

$B = 2$

$D = A^{1/B} = \seqsplit{45970326344906740713427350692252807528602033916808397812264455041567960827591052996772483599768476236004358388365770278914227572357628639076590737735330984}$

При вычислении квадратного корня $w_3$ первая строка таблицы остаётся пустой, что свидетельствует об успехе факторизации. 

Вычисляем $p = t_3 + sqrt(w3)$

$A = t_3$

$B = sqrt(w_3)$

$D = A + B = p = \seqsplit{291411547907966097437475668012142221876298569079318524242183274281819560023886760140878408980756408066198439646241201452287857402991893383345901903898208711989745179623129930067121350232874745997609164975272204699142494582149842464324535497403195585486887640432238367427980482608077614212347781251141637890811}$

$q = t_3 – sqrt(w_3) = \seqsplit{291411547907966097437475668012142221876298569079318524242183274281819560023886760140878408980756408066198439646241201452287857402991893383345901903898208620049092489809648503212419965727259688793541331358476580170232411446228187282218541952435996048534415631715461635887422654152932898955069628069666167228843}$

Вычисляем $Phi(N) = (p – 1)(q – 1)$
 
$A = p – 1$

$B = q – 1$

$D = A \cdot B = Phi(N) =
\seqsplit{84920690254116819980017474476393118148268821013112888110434002286862060838164293058912544765163219640705110180259944198609255581868934872999810160228209843869386758250683376679520802497046978250461593283557666939606457018130540475876808030500520994415313016543600381848593348673332775443216238563477836798691960893675875031823223628092795668836452333340804926965087710240446079720189881088192580507465749772625657400702200086503810862751233292062585395066605419660777682254623706509828081855755165364997931513379724919689682499186894867679911257965595529583820488963258525910077568678090925370204561708451923478742020}$

Вычисляем $d$, как обратный экспоненте $e$: 

$A = e$

$B = –1$

$C = Phi(N)$

$D = A^B mod C = d = 
\seqsplit{14468764328633144155555109499723048929104078633164036376603635187206837384121210881403998629848933003953520402126579368026163861160975235709280172271790276515412883432200461401418780412189686932343495634382579225199261187753153611244710949843282119225970576521909877752964718404783675696088750109316199582560011349175418354488130293824085255132412125676788041245261203556492497992299154482209632618716330315411786545777123411725136642180000800445064214772179714428926431844908470748019344417829990790047849661622238971572240190394090966509424270830878258758528083640192152641232901045005421446363836778234700836429253}$

Производим дешифрацию шифрблока $С$: 

$A = C$

$B = d$

$C = N$

$D = A^B mod C = 
\seqsplit{1827661253841309593956183709238468584479056570140676378135270478122258697254998218454942377216470497789727245323580130409751689532533434536074118414379455015258874379147688413833138020086598156330400963215837811004084604515699798298569426558254332104194601935895422329713905722651724343445137060326248770190369550013034236337747150886965134068936608719662246245131174238556398414991606762958162799917576314491449753755973034160550341607206641040036827705128812569462491240420121231156032061778493888517417629187982460589379145535512730513566800235525890007820127375039523278283105089457820683207623444271262303}$

Ответ: "технологическая, производственная, финансово-экономическая или иная информация (в том числе составляющая секреты производства (ноу-хау), которая имеет действительную или потенциальную коммерческую ценность  в силу  неизвестности ее третьим лицам".



