\section{Структура сети Фейстеля}

В 1971 году Хорст Фейстель разработал два устройства, реализовавшие различные алгоритмы шифрования, названные затем общим название «Люцифер». В одной из этих устройств он использовал схему, которую впоследствии назвали сетью Фейстеля. Эта сеть представляет собой определённую многократно итерированную (повторяющуюся) структуру, которою называют ячейкой Фейстеля.

\begin{figure}[H]
	\centering
	\includegraphics[width=0.7\linewidth]{"../Лабораторные по Ринджиндайлу/img/feistel"}
	\caption{Схема сети Фейстеля}
\end{figure}

Принцип работы:

\begin{enumerate}
	\item Исходные данные разбиваются на блоки фиксированной длины (как правило кратно степени двойки — 64 бит, 128 бит). В случае если длина блока исходных данных меньше длины разрядности шифра, то блок дополняется каким-либо заранее известным образом.

	\item Блок делится на два равных подблока — «левый» $L_0$ и «правый» $R_0$. В случае 64-битной разрядности — на два блока с длиной 32 бита каждый.

	\item «Левый подблок» $L_0$ видоизменяется функцией итерации $F(L_0, P_0)$ в зависимости от ключа $P_0$, после чего он складывается по модулю 2 (XOR) с «правым подблоком» $R_0$.

	\item Результат сложения присваивается новому левому подблоку $L_1$, который становится левой половиной входных данных для следующего раунда, а «левый подблок» $L_0$ присваивается без изменений новому правому подблоку $R_1$, который становится правой половиной.

	\item Эта операция повторяется $n-1$ раз, при этом при переходе от одного этапа к другому меняются раундовые ключи ($P_0$, $P_1$, $P_2$ и т.д.), где $n$ — количество раундов для используемого алгоритма.
\end{enumerate}

Процесс расшифрования аналогичен процессу шифрования за исключением того, что раундовые ключи используются в обратном порядке.
