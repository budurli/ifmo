
Вычисляем 2G:

            \[
            \lambda = \frac{ 3 \cdot 0^2 + (-1) }{2 \cdot 1} = \frac{-1}{2} = -1 \cdot 2^{-1} = 750 \cdot 376\mod{751} = 375\mod{751}
            \]
            

        \[
        x = 375^2 - 0 - 0 \mod{751} = 188\mod{751}
        \]
        

        \[
        y = 375 \cdot (0 - 188) - 1\mod{751} = 93\mod{751}
        \]
        
\textbf{2G = (188, 93)}

Вычисляем 3G:

            \[
            \lambda = \frac{1-93}{0-188} = \frac{-92}{-188} = 92 \cdot 188^{-1}\mod{751} = 92 \cdot 4\mod{751}
            \]
            

        \[
        x = 368^2 - 188 - 0 \mod{751} = 56\mod{751}
        \]
        

        \[
        y = 368 \cdot (188 - 56) - 93\mod{751} = 419\mod{751}
        \]
        
\textbf{3G = (56, 419)}

Вычисляем 4G:

            \[
            \lambda = \frac{ 3 \cdot 188^2 + (-1) }{2 \cdot 93} = \frac{106031}{186} = 106031 \cdot 186^{-1} = 140 \cdot 214\mod{751} = 671\mod{751}
            \]
            

        \[
        x = 671^2 - 188 - 188 \mod{751} = 16\mod{751}
        \]
        

        \[
        y = 671 \cdot (188 - 16) - 93\mod{751} = 416\mod{751}
        \]
        
\textbf{4G = (16, 416)}

Вычисляем 5G:

            \[
            \lambda = \frac{93-419}{188-56} = \frac{-326}{132} = -326 \cdot 132^{-1}\mod{751} = 425 \cdot 165\mod{751}
            \]
            

        \[
        x = 282^2 - 56 - 188 \mod{751} = 425\mod{751}
        \]
        

        \[
        y = 282 \cdot (56 - 425) - 419\mod{751} = 663\mod{751}
        \]
        
\textbf{5G = (425, 663)}

Вычисляем 6G:

            \[
            \lambda = \frac{ 3 \cdot 56^2 + (-1) }{2 \cdot 419} = \frac{9407}{838} = 9407 \cdot 838^{-1} = 395 \cdot 587\mod{751} = 557\mod{751}
            \]
            

        \[
        x = 557^2 - 56 - 56 \mod{751} = 725\mod{751}
        \]
        

        \[
        y = 557 \cdot (56 - 725) - 419\mod{751} = 195\mod{751}
        \]
        
\textbf{6G = (725, 195)}

Вычисляем 7G:

            \[
            \lambda = \frac{419-416}{56-16} = \frac{3}{40} = 3 \cdot 40^{-1}\mod{751} = 3 \cdot 169\mod{751}
            \]
            

        \[
        x = 507^2 - 16 - 56 \mod{751} = 135\mod{751}
        \]
        

        \[
        y = 507 \cdot (16 - 135) - 416\mod{751} = 82\mod{751}
        \]
        
\textbf{7G = (135, 82)}

Вычисляем 8G:

            \[
            \lambda = \frac{ 3 \cdot 16^2 + (-1) }{2 \cdot 416} = \frac{767}{832} = 767 \cdot 832^{-1} = 16 \cdot 102\mod{751} = 130\mod{751}
            \]
            

        \[
        x = 130^2 - 16 - 16 \mod{751} = 346\mod{751}
        \]
        

        \[
        y = 130 \cdot (16 - 346) - 416\mod{751} = 242\mod{751}
        \]
        
\textbf{8G = (346, 242)}

Вычисляем 9G:

            \[
            \lambda = \frac{416-663}{16-425} = \frac{-247}{-409} = 247 \cdot 409^{-1}\mod{751} = 247 \cdot 213\mod{751}
            \]
            

        \[
        x = 41^2 - 425 - 16 \mod{751} = 489\mod{751}
        \]
        

        \[
        y = 41 \cdot (425 - 489) - 663\mod{751} = 468\mod{751}
        \]
        
\textbf{9G = (489, 468)}

Вычисляем 10G:

            \[
            \lambda = \frac{ 3 \cdot 425^2 + (-1) }{2 \cdot 663} = \frac{541874}{1326} = 541874 \cdot 1326^{-1} = 403 \cdot 64\mod{751} = 258\mod{751}
            \]
            

        \[
        x = 258^2 - 425 - 425 \mod{751} = 377\mod{751}
        \]
        

        \[
        y = 258 \cdot (425 - 377) - 663\mod{751} = 456\mod{751}
        \]
        
\textbf{10G = (377, 456)}

Вычисляем 11G:

            \[
            \lambda = \frac{1-456}{0-377} = \frac{-455}{-377} = 455 \cdot 377^{-1}\mod{751} = 455 \cdot 251\mod{751}
            \]
            

        \[
        x = 53^2 - 377 - 0 \mod{751} = 179\mod{751}
        \]
        

        \[
        y = 53 \cdot (377 - 179) - 456\mod{751} = 275\mod{751}
        \]
        
\textbf{11G = (179, 275)}

Вычисляем 12G:

            \[
            \lambda = \frac{ 3 \cdot 725^2 + (-1) }{2 \cdot 195} = \frac{1576874}{390} = 1576874 \cdot 390^{-1} = 525 \cdot 518\mod{751} = 88\mod{751}
            \]
            

        \[
        x = 88^2 - 725 - 725 \mod{751} = 286\mod{751}
        \]
        

        \[
        y = 88 \cdot (725 - 286) - 195\mod{751} = 136\mod{751}
        \]
        
\textbf{12G = (286, 136)}

Вычисляем 13G:

            \[
            \lambda = \frac{195-82}{725-135} = \frac{113}{590} = 113 \cdot 590^{-1}\mod{751} = 113 \cdot 737\mod{751}
            \]
            

        \[
        x = 671^2 - 135 - 725 \mod{751} = 283\mod{751}
        \]
        

        \[
        y = 671 \cdot (135 - 283) - 82\mod{751} = 493\mod{751}
        \]
        
\textbf{13G = (283, 493)}

Вычисляем 14G:

            \[
            \lambda = \frac{ 3 \cdot 135^2 + (-1) }{2 \cdot 82} = \frac{54674}{164} = 54674 \cdot 164^{-1} = 602 \cdot 664\mod{751} = 196\mod{751}
            \]
            

        \[
        x = 196^2 - 135 - 135 \mod{751} = 596\mod{751}
        \]
        

        \[
        y = 196 \cdot (135 - 596) - 82\mod{751} = 433\mod{751}
        \]
        
\textbf{14G = (596, 433)}

Вычисляем 15G:

            \[
            \lambda = \frac{82-242}{135-346} = \frac{-160}{-211} = 160 \cdot 211^{-1}\mod{751} = 160 \cdot 210\mod{751}
            \]
            

        \[
        x = 556^2 - 346 - 135 \mod{751} = 745\mod{751}
        \]
        

        \[
        y = 556 \cdot (346 - 745) - 242\mod{751} = 210\mod{751}
        \]
        
\textbf{15G = (745, 210)}

Вычисляем 16G:

            \[
            \lambda = \frac{ 3 \cdot 346^2 + (-1) }{2 \cdot 242} = \frac{359147}{484} = 359147 \cdot 484^{-1} = 169 \cdot 45\mod{751} = 95\mod{751}
            \]
            

        \[
        x = 95^2 - 346 - 346 \mod{751} = 72\mod{751}
        \]
        

        \[
        y = 95 \cdot (346 - 72) - 242\mod{751} = 254\mod{751}
        \]
        
\textbf{16G = (72, 254)}

Вычисляем 17G:

            \[
            \lambda = \frac{242-468}{346-489} = \frac{-226}{-143} = 226 \cdot 143^{-1}\mod{751} = 226 \cdot 730\mod{751}
            \]
            

        \[
        x = 511^2 - 489 - 346 \mod{751} = 440\mod{751}
        \]
        

        \[
        y = 511 \cdot (489 - 440) - 468\mod{751} = 539\mod{751}
        \]
        
\textbf{17G = (440, 539)}

Вычисляем 18G:

            \[
            \lambda = \frac{ 3 \cdot 489^2 + (-1) }{2 \cdot 468} = \frac{717362}{936} = 717362 \cdot 936^{-1} = 157 \cdot 341\mod{751} = 216\mod{751}
            \]
            

        \[
        x = 216^2 - 489 - 489 \mod{751} = 618\mod{751}
        \]
        

        \[
        y = 216 \cdot (489 - 618) - 468\mod{751} = 206\mod{751}
        \]
        
\textbf{18G = (618, 206)}

Вычисляем 19G:

            \[
            \lambda = \frac{468-456}{489-377} = \frac{12}{112} = 12 \cdot 112^{-1}\mod{751} = 12 \cdot 114\mod{751}
            \]
            

        \[
        x = 617^2 - 377 - 489 \mod{751} = 568\mod{751}
        \]
        

        \[
        y = 617 \cdot (377 - 568) - 456\mod{751} = 355\mod{751}
        \]
        
\textbf{19G = (568, 355)}

 Вычисляем $Pm+kPb$ для каждой буквы в слове.
