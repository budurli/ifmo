\section{Шифр двойной перестановки}

Дан шифр-текст: \textbf{А\_ЛТАВЙООЛСО\_ТВ\_ШЕЕНЕСТ\_Ь}

Текст содержит 25 символов, что позволяет записать его в квадратную матрицу 5х5. известно, что шифрование производилось сначала по столбцам, а затем по строкам, следовательно, расшифрование следует проводить тем же способом.

\begin{table}[H]
	\centering
	\begin{tabular}{|c|c|c|c|c|}
		\hline
		А  & \_ & Л  & Т  & А \\ \hline
		В  & Й  & О  & О  & Л \\ \hline
		С  & О  & \_ & Т  & В \\ \hline
		\_ & Ш  & Е  & Е  & Н \\ \hline
		Е  & С  & Т  & \_ & Ь \\ \hline
	\end{tabular}
	\caption{Шифр-текст в виде матрицы}
\end{table}

Для оптимизации скорости выполнения задания можно проверить все комбинации букв только во второй строке: ВЙ - х, ВО - 58, ВЛ - 10, ЙВ - 4, ЙО - 0, ЙЛ - 1, ОВ - 84, ОЙ - 29, ОО - 9, ОЛ - 41, ЛВ - 1, ЛЙ - х, ЛО - 30.	

Подходит сочетание ЛОВОЙ.

Получим:

\begin{table}[H]
	\centering
	\begin{tabular}{|c|c|c|c|c|}
		\hline
		А & Л  & А  & Т  & \_ \\ \hline
		Л & О  & В  & О  & Й  \\ \hline
		В & \_ & С  & Т  & О  \\ \hline
		Н & Е  & \_ & Е  & Ш  \\ \hline
		Ь & Т  & Е  & \_ & С  \\ \hline
	\end{tabular}
\end{table}

Очевидная перестановка:
\begin{table}[H]
	\centering
	\begin{tabular}{|c|c|c|c|c|}
		\hline
		В & \_ & С  & Т  & О  \\ \hline
		Л & О  & В  & О  & Й  \\ \hline
		Н & Е  & \_ & Е  & Ш  \\ \hline
		Ь & Т  & Е  & \_ & С  \\ \hline
		А & Л  & А  & Т  & \_ \\ \hline
	\end{tabular}
\end{table}


Получаем осмысленный текст: \textbf{В\_СТОЛОВОЙНЕ\_ЕШЬТЕ\_САЛАТ\_}
