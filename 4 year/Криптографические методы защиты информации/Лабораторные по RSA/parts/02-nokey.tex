\section{Атака на алгоритм шифрования RSA методом бесключевого чтения}

\textbf{Цель работы}: изучить атаку на алгоритм шифрования RSA посредством метода бесключевого чтения.

Исходные данные: $N = 516439217617; e_1 = 1206433; e_2 = 1141277; 
C_1 = \seqsplit{%
	400408320444
	241545246801
	282223079755
	490328978748
	350509811006
	142356755075
	109547314116
	414823859933
	330990395685
	377471732609
	44017319588
	499241372980
	171071879560}; C_2 = \seqsplit{%
	374984721363
	438491303024
	498951362977
	218681974856
	365827206348
	175049781656
	359111505460
	297734746741
	96963152197
	362138584797
	102758207364
	37817394150
	120430068125}$.

\begin{itemize}
\item Решаем уравнение $e_1 \cdot r – e_2 \cdot s = \pm 1$. Для этого в поле A помещаем значение $e_1$, в поле B – значение $e_2$. Нажимаем кнопку «A $\cdot$ D – B $\cdot$ C = N»; C = s = -406030; D = r = 286243
\item $c_1 = 400408320444$, $c_2=94559770883$
\item Производим дешифрацию: $c_1$ возводим в степень $r$, а $c_2$ – в степень $–s$ по модулю $N$, тогда $c_1 ^ r = 230407699699$, $c_2 ^{–s} = 32354372535$.
\item После этого результаты перемножаем и получаем, что $m ^ {e1 \cdot r – e2 \cdot s} = \seqsplit{7454696550993853366965}$. Далее берём модуль от полученного значения:  $m ^ {(e1 \cdot r – e2 \cdot s)} mod N) = 200458576327$ и преобразуем в текст <<>>
\item $c_1 = 241545246801$, $c_2=144847640787$
\item Производим дешифрацию: $c_1$ возводим в степень $r$, а $c_2$ – в степень $–s$ по модулю $N$, тогда $c_1 ^ r = 51618928882$, $c_2 ^{–s} = 27509250912$.
\item После этого результаты перемножаем и получаем, что $m ^ {e1 \cdot r – e2 \cdot s} = \seqsplit{1419998066423621640384}$. Далее берём модуль от полученного значения:  $m ^ {(e1 \cdot r – e2 \cdot s)} mod N) = 82130033820$ и преобразуем в текст <<>>
\item $c_1 = 282223079755$, $c_2=236499554090$
\item Производим дешифрацию: $c_1$ возводим в степень $r$, а $c_2$ – в степень $–s$ по модулю $N$, тогда $c_1 ^ r = 259972599441$, $c_2 ^{–s} = 8867154266$.
\item После этого результаты перемножаем и получаем, что $m ^ {e1 \cdot r – e2 \cdot s} = \seqsplit{2305217144176372365306}$. Далее берём модуль от полученного значения:  $m ^ {(e1 \cdot r – e2 \cdot s)} mod N) = 117686786725$ и преобразуем в текст <<>>
\item $c_1 = 490328978748$, $c_2=91691946714$
\item Производим дешифрацию: $c_1$ возводим в степень $r$, а $c_2$ – в степень $–s$ по модулю $N$, тогда $c_1 ^ r = 318730996923$, $c_2 ^{–s} = 306230064232$.
\item После этого результаты перемножаем и получаем, что $m ^ {e1 \cdot r – e2 \cdot s} = \seqsplit{97605013660459684358136}$. Далее берём модуль от полученного значения:  $m ^ {(e1 \cdot r – e2 \cdot s)} mod N) = 315058382091$ и преобразуем в текст <<>>
\item $c_1 = 350509811006$, $c_2=195676236846$
\item Производим дешифрацию: $c_1$ возводим в степень $r$, а $c_2$ – в степень $–s$ по модулю $N$, тогда $c_1 ^ r = 8257228465$, $c_2 ^{–s} = 292435110568$.
\item После этого результаты перемножаем и получаем, что $m ^ {e1 \cdot r – e2 \cdot s} = \seqsplit{2414703519147511918120}$. Далее берём модуль от полученного значения:  $m ^ {(e1 \cdot r – e2 \cdot s)} mod N) = 100777754363$ и преобразуем в текст <<>>
\item $c_1 = 142356755075$, $c_2=105890375451$
\item Производим дешифрацию: $c_1$ возводим в степень $r$, а $c_2$ – в степень $–s$ по модулю $N$, тогда $c_1 ^ r = 133217995591$, $c_2 ^{–s} = 125953470039$.
\item После этого результаты перемножаем и получаем, что $m ^ {e1 \cdot r – e2 \cdot s} = \seqsplit{16779268816326652598049}$. Далее берём модуль от полученного значения:  $m ^ {(e1 \cdot r – e2 \cdot s)} mod N) = 276740551605$ и преобразуем в текст <<>>
\item $c_1 = 109547314116$, $c_2=248047563144$
\item Производим дешифрацию: $c_1$ возводим в степень $r$, а $c_2$ – в степень $–s$ по модулю $N$, тогда $c_1 ^ r = 237537182768$, $c_2 ^{–s} = 342278387913$.
\item После этого результаты перемножаем и получаем, что $m ^ {e1 \cdot r – e2 \cdot s} = \seqsplit{81303843987226683083184}$. Далее берём модуль от полученного значения:  $m ^ {(e1 \cdot r – e2 \cdot s)} mod N) = 283204328507$ и преобразуем в текст <<>>
\item $c_1 = 414823859933$, $c_2=134557356194$
\item Производим дешифрацию: $c_1$ возводим в степень $r$, а $c_2$ – в степень $–s$ по модулю $N$, тогда $c_1 ^ r = 277612889360$, $c_2 ^{–s} = 125487085349$.
\item После этого результаты перемножаем и получаем, что $m ^ {e1 \cdot r – e2 \cdot s} = \seqsplit{34836832341100813986640}$. Далее берём модуль от полученного значения:  $m ^ {(e1 \cdot r – e2 \cdot s)} mod N) = 190359328241$ и преобразуем в текст <<>>
\item $c_1 = 330990395685$, $c_2=223041604801$
\item Производим дешифрацию: $c_1$ возводим в степень $r$, а $c_2$ – в степень $–s$ по модулю $N$, тогда $c_1 ^ r = 47552794481$, $c_2 ^{–s} = 179282854294$.
\item После этого результаты перемножаем и получаем, что $m ^ {e1 \cdot r – e2 \cdot s} = \seqsplit{8525400724209650351414}$. Далее берём модуль от полученного значения:  $m ^ {(e1 \cdot r – e2 \cdot s)} mod N) = 71874538543$ и преобразуем в текст <<>>
\item $c_1 = 377471732609$, $c_2=138272971125$
\item Производим дешифрацию: $c_1$ возводим в степень $r$, а $c_2$ – в степень $–s$ по модулю $N$, тогда $c_1 ^ r = 172757639651$, $c_2 ^{–s} = 10278982505$.
\item После этого результаты перемножаем и получаем, что $m ^ {e1 \cdot r – e2 \cdot s} = \seqsplit{1775772755577723305755}$. Далее берём модуль от полученного значения:  $m ^ {(e1 \cdot r – e2 \cdot s)} mod N) = 307737033561$ и преобразуем в текст <<>>
\item $c_1 = 44017319588$, $c_2=249978808424$
\item Производим дешифрацию: $c_1$ возводим в степень $r$, а $c_2$ – в степень $–s$ по модулю $N$, тогда $c_1 ^ r = 352291575112$, $c_2 ^{–s} = 269888526727$.
\item После этого результаты перемножаем и получаем, что $m ^ {e1 \cdot r – e2 \cdot s} = \seqsplit{95079454185311940018424}$. Далее берём модуль от полученного значения:  $m ^ {(e1 \cdot r – e2 \cdot s)} mod N) = 201830183707$ и преобразуем в текст <<>>
\item $c_1 = 499241372980$, $c_2=344974502483$
\item Производим дешифрацию: $c_1$ возводим в степень $r$, а $c_2$ – в степень $–s$ по модулю $N$, тогда $c_1 ^ r = 17047766454$, $c_2 ^{–s} = 211271016734$.
\item После этого результаты перемножаем и получаем, что $m ^ {e1 \cdot r – e2 \cdot s} = \seqsplit{3601698951780357841236}$. Далее берём модуль от полученного значения:  $m ^ {(e1 \cdot r – e2 \cdot s)} mod N) = 230705485731$ и преобразуем в текст <<>>
\item $c_1 = 171071879560$, $c_2=108413221760$
\item Производим дешифрацию: $c_1$ возводим в степень $r$, а $c_2$ – в степень $–s$ по модулю $N$, тогда $c_1 ^ r = 37543645247$, $c_2 ^{–s} = 335279611747$.
\item После этого результаты перемножаем и получаем, что $m ^ {e1 \cdot r – e2 \cdot s} = \seqsplit{12587618801981261916509}$. Далее берём модуль от полученного значения:  $m ^ {(e1 \cdot r – e2 \cdot s)} mod N) = 201615547888$ и преобразуем в текст <<>>
\end{itemize}
