\section{Ход работы}

\subsection{Формирование сообщения}

\begin{table}[H]
	\centering
	\begin{tabular}{|c|c|}
		\hline
		Исходное сообщение       & Смирнов М.Г.                                                                                                \\ \hline
		В шестнадцатеричном коде & D1 EC E8 F0 ED EE E2 20 CC 2E C3 2E                                                                         \\ \hline
		В двоичном коде          & \parbox[c][3cm]{9cm}{11010001 11101100 11101000 11110000 11101101 11101110 11100010 00100000 11001100 00101110 11000011 00101110} \\ \hline
		Длина сообщения          & 12 байт (96 бит)                                                                                            \\ \hline
	\end{tabular}
\end{table}

\subsection{Физическое кодирование исходного сообщения}

Потенциальный код без возврата к нулю (NRZ):

\begin{figure}[H]
	\centering
	
	\begin{tikztimingtable}[timing/slope=0, scale=1]
	NRZ          & HHLHLLLHHHHLHHLLHHHLHLLLHHHHLLLL \\
	\extracode
	\makeatletter
	\begin{pgfonlayer}{background}
		\vertlines[help lines, black]{}
		\horlines[black]{}
		\foreach [count=\x] \b in {1,1,0,1,0,0,0,1,1,1,1,0,1,1,0,0,1,1,1,0,1,0,0,0,1,1,1,1,0,0,0,0} {
			\node [above,font=\sffamily\bfseries\tiny,inner ysep=0pt] at (\x-.5,-1.5) {\b};
		}
	\end{pgfonlayer}
\end{tikztimingtable}
\end{figure}

\[
f_\text{в} = \frac{C}{2} = 500 \text{ мГц}
\]

\[
f_\text{н} = \frac{C}{8} = 125 \text{ мГц}
\]

\[
S = f_\text{в} - f_\text{н} = 375  \text{ мГц}
\]

\[
f_\text{ср} = \left( 5 \cdot f_0 + 6 \cdot \frac{f_0}{2} + 9 \cdot \frac{f_0}{3} + 12 \cdot \frac{f_0}{4} \right) / 32 = \frac{14 \cdot f_0}{32} = 0.4375 \cdot f_0 = 218.75 \text{ мГц}
\]

Биполярный импульсный код (RZ):

\begin{figure}[H]
	\centering
	
	\begin{tikztimingtable}[timing/slope=0, scale=1]
		RZ           & hzhzlzhzlzlzlzhzhzhzhzlzhzhzlzlzhzhzhzlzhzlzlzlzhzhzhzhzlzlzlzlz\\
		\extracode
		\makeatletter
		\begin{pgfonlayer}{background}
			\vertlines[help lines, black]{}
			\horlines[black,yshift=2mm]{}
			\foreach [count=\x] \b in {1,1,0,1,0,0,0,1,1,1,1,0,1,1,0,0,1,1,1,0,1,0,0,0,1,1,1,1,0,0,0,0} {
				\node [above,font=\sffamily\bfseries\tiny,inner ysep=0pt] at (\x-.5,-1.5) {\b};
			}
		\end{pgfonlayer}
	\end{tikztimingtable}
	
\end{figure}


\[
f_\text{в} = C = 1000 \text{ мГц}
\]

\[
f_\text{н} = \frac{C}{4} = 250 \text{ мГц}
\]

\[
S = f_\text{в} - f_\text{н} = 375  \text{ мГц}
\]

\[
f_\text{ср} = 675 \text{ мГц}
\]

Биполярное кодирование с чередующейся инверсией (AMI):

\begin{figure}[H]
	\centering
	
	\begin{tikztimingtable}[timing/slope=0, scale=1]
		AMI & HLZHZZZLHLHZLHZZLHLZHZZZLHLHZZZZ \\
		\extracode
		\makeatletter
		\begin{pgfonlayer}{background}
			\vertlines[help lines, black]{}
			\horlines[black,yshift=2mm]{}
			\foreach [count=\x] \b in {1,1,0,1,0,0,0,1,1,1,1,0,1,1,0,0,1,1,1,0,1,0,0,0,1,1,1,1,0,0,0,0} {
				\node [above,font=\sffamily\bfseries\tiny,inner ysep=0pt] at (\x-.5,-1.5) {\b};
			}
		\end{pgfonlayer}
	\end{tikztimingtable}
	
\end{figure}

\[
f_\text{в} = \frac{C}{2} = 500 \text{ мГц}
\]

\[
f_\text{н} = \frac{C}{4} = 250 \text{ мГц}
\]

\[
S = f_\text{в} - f_\text{н} = 250  \text{ мГц}
\]

\[
f_\text{ср} = 360 \text{ мГц}
\]

Манчестерский код:

\begin{figure}[H]
	\centering
	\begin{tikztimingtable}[scale=1,timing/.cd,
		c/dual arrows,c/arrow tip=latex,
		c/arrow pos=.7,
		metachar={v}{[timing/c/no arrows]c[timing/c/dual arrows]},
		slope=0]
		\shortstack[l]{M} 
		&zhzhhzzhhzhzhzzhzhzhzhhzzhzhhzhzzhzhzhhzzhhzhzhzzhzhzhzhhzhzhzhz\\
		\extracode
		\begin{pgfonlayer}{background}
			\vertlines[help lines,brown]{}
			\foreach [count=\x] \b in {1,1,0,1,0,0,0,1,1,1,1,0,1,1,0,0,1,1,1,0,1,0,0,0,1,1,1,1,0,0,0,0} {
				\node [below,font=\sffamily\bfseries\tiny,inner ysep=2pt] at (\x-.5,0) {\b};
			}
		\end{pgfonlayer}
	\end{tikztimingtable}
\end{figure}

\[
f_\text{в} = C = 1000 \text{ мГц}
\]

\[
f_\text{н} = \frac{C}{2} = 500 \text{ мГц}
\]

\[
S = f_\text{в} - f_\text{н} = 500  \text{ мГц}
\]

\[
f_\text{ср} = 765,625 \text{ мГц}
\]

\begin{table}[H]
	\centering
	\begin{tabular}{|c|c|c|c|c|}
		\hline
		& NRZ & RZ  & AMI & Манчестерский код \\ \hline
		Ширина спектра, МГц                & 334 & 750 & 250 & 500               \\ \hline
		Отсутствие постоянной составляющей & -   & +   & +   & +                 \\ \hline
		Самосинхронизация                  & -   & +   & +   & +                 \\ \hline
		Количество уровней                 & 2   & 3   & 3   & 2                 \\ \hline
		Распознавание ошибок               & -   & -   & +   & -                 \\ \hline
	\end{tabular}
\end{table}

На основании таблицы можно сделать вывод, что наиболее подходящими являются AMI и Манчестерский код.

\subsection{Логическое (избыточное) кодирование исходного сообщения}

Исходное сообщение: \textbf{1101 0001 1110 1100 1110 1000 1111 0000 1110 1101 1110 1110 1110 0010 0010 0000 1100 1100 0010 1110 1100 0011 0010 1110}.

4b/5b: \textbf{1101 1010 0111 1001 1010 1110 0100 1011 1011 1110 1110 0110 1111 1001 1100 1110 0101 0010 1001 1110 1101 0110 1010 1001 1100 1101 0101 0110 1001 1100}.

В шестнадцатеричном коде: \textbf{DA 79 AE 4B BE E6 F9 CE 52 9E D6 A9 CD 56 9C}.

Длина сообщения: 120 бит.

Биполярное кодирование с чередующейся инверсией (AMI):

\begin{figure}[H]
	\centering
	
	\begin{tikztimingtable}[timing/slope=0, scale=1]
		AMI & HLZHLZHZZLHLHZZLHZLZHLHZZLZZHZLH \\
		\extracode
		\makeatletter
		\begin{pgfonlayer}{background}
			\vertlines[help lines, black]{}
			\horlines[black,yshift=2mm]{}
			\foreach [count=\x] \b in {1,1,0,1,1,0,1,0,0,1,1,1,1,0,0,1,1,0,1,0,1,1,1,0,0,1,0,0,1,0,1,1} {
				\node [above,font=\sffamily\bfseries\tiny,inner ysep=0pt] at (\x-.5,-1.5) {\b};
			}
		\end{pgfonlayer}
	\end{tikztimingtable}
	
\end{figure}

\[
f_\text{в} = \frac{C}{2} = 500 \text{ мГц}
\]

\[
f_\text{н} = \frac{C}{4} = 250 \text{ мГц}
\]

\[
S = f_\text{в} - f_\text{н} = 250  \text{ мГц}
\]

\[
f_\text{ср} = 360 \text{ мГц}
\]

Манчестерский код:

\begin{figure}[H]
	\centering
	\begin{tikztimingtable}[scale=1,timing/.cd,
		c/dual arrows,c/arrow tip=latex,
		c/arrow pos=.7,
		metachar={v}{[timing/c/no arrows]c[timing/c/dual arrows]},
		slope=0]
		\shortstack[l]{M} 
		&zhzhhzzhhzhzhzzhzhzhzhhzzhzhhzhzzhzhzhhzzhhzhzhzzhzhzhzhhzhzhzhz\\
		\extracode
		\begin{pgfonlayer}{background}
			\vertlines[help lines,brown]{}
			\foreach [count=\x] \b in {1,1,0,1,0,0,0,1,1,1,1,0,1,1,0,0,1,1,1,0,1,0,0,0,1,1,1,1,0,0,0,0} {
				\node [below,font=\sffamily\bfseries\tiny,inner ysep=2pt] at (\x-.5,0) {\b};
			}
		\end{pgfonlayer}
	\end{tikztimingtable}
\end{figure}

\[
f_\text{в} = C = 1000 \text{ мГц}
\]

\[
f_\text{н} = \frac{C}{2} = 500 \text{ мГц}
\]

\[
S = f_\text{в} - f_\text{н} = 500  \text{ мГц}
\]

\[
f_\text{ср} = 725 \text{ мГц}
\]

В биполярном кодирование с чередующейся инверсией средняя частота и ширина спектра гораздо ниже, чем в манчестерском коде, поэтому оно более оптимально

\subsection{Скремблирование исходного сообщения}

Будем использовать выражение $B_i = A_i \oplus B_{i-3} \oplus B_{i-5}$

Исходное сообщение: \textbf{1101 0001 1110 1100 1110 1000 1111 0000 1110 1101 1110 1110 1110 0010 0010 0000 1100 1100 0010 1110 1100 0011 0010 1110}.

Скремблированное сообщение: \textbf{1100 1101 0101 1110 0001 0010 1010 1001 0111 0111 0011 1100 0100 1001 0111 0101 1101 0011 1100 1010 0110 1101 0101 0010}.

Биполярное кодирование с чередующейся инверсией (AMI):

\begin{figure}[H]
	\centering
	
	\begin{tikztimingtable}[timing/slope=0, scale=1]
		AMI & HLZZHLZHZLZHLHLZZZZHZZLZHZLZHZZL \\
		\extracode
		\makeatletter
		\begin{pgfonlayer}{background}
			\vertlines[help lines, black]{}
			\horlines[black,yshift=2mm]{}
			\foreach [count=\x] \b in {1,1,0,0,1,1,0,1,0,1,0,1,1,1,1,0,0,0,0,1,0,0,1,0,1,0,1,0,1,0,0,1} {
				\node [above,font=\sffamily\bfseries\tiny,inner ysep=0pt] at (\x-.5,-1.5) {\b};
			}
		\end{pgfonlayer}
	\end{tikztimingtable}
	
\end{figure}

\[
f_\text{в} = \frac{C}{2} = 500 \text{ мГц}
\]

\[
f_\text{н} = \frac{C}{4} = 250 \text{ мГц}
\]

\[
S = f_\text{в} - f_\text{н} = 250  \text{ мГц}
\]

\[
f_\text{ср} = 360 \text{ мГц}
\]

Манчестерский код:

\begin{figure}[H]
	\centering
	\begin{tikztimingtable}[scale=1,timing/.cd,
		c/dual arrows,c/arrow tip=latex,
		c/arrow pos=.7,
		metachar={v}{[timing/c/no arrows]c[timing/c/dual arrows]},
		slope=0]
		\shortstack[l]{M}
		& zhzhzlzlzhzhzlzhzlzhzlzhzhzhzhzlzlzlzlzhzlzlzhzlzhzlzhzlzhzlzlzh \\
		\extracode
		\begin{pgfonlayer}{background}
			\vertlines[help lines,brown]{}
			\foreach [count=\x] \b in {1,1,0,0,1,1,0,1,0,1,0,1,1,1,1,0,0,0,0,1,0,0,1,0,1,0,1,0,1,0,0,1} {
				\node [below,font=\sffamily\bfseries\tiny,inner ysep=2pt] at (\x-.5,0) {\b};
			}
		\end{pgfonlayer}
	\end{tikztimingtable}
\end{figure}
\[
f_\text{в} = C = 1000 \text{ мГц}
\]

\[
f_\text{н} = \frac{C}{2} = 500 \text{ мГц}
\]

\[
S = f_\text{в} - f_\text{н} = 500  \text{ мГц}
\]

\[
f_\text{ср} = 725 \text{ мГц}
\]

В биполярном кодирование с чередующейся инверсией средняя частота и ширина спектра ниже, чем в манчестерском коде, поэтому оно наиболее оптимально

\subsection{Сравнительный анализ результатов кодирования}

Оптимальным способом кодирования для исходного сообщения является биполярное кодирование с чередующейся инверсией.

В AMI ширина спектра ниже, чем у других способов кодирования, присутствует самосинхронизация и возможность распознавания ошибок. При этом у него три уровня кодирования, что увеличивает затраты на мощность и может появиться постоянная составляющая.
