\section{Ход работы}

Имеются следующие подразделения:

\begin{itemize}
	\item Подразделение эксплуатации ТС
	\item Подразделение, обеспечивающее ИБ
	\item Подразделение, устанавливающее СЗИ
	\item Подразделение мониторинга и контроля безопасности
\end{itemize}

Составим организационную схему:

\begin{figure}[H]
	\centering
\begin{forest}
	for tree={
		draw,
		align=center
	},
	forked edges,
	[Департамент ИБ \\ \textbf{Директор по ИБ}
		[Отдел эксплуатации ТС \\ \textbf{Начальник отдела}]
		[Отдел обеспечения ИБ \\ \textbf{Начальник отдела}
			[Группа установки СЗИ \\ \textbf{Ведущий специалист}]
			[Группа мониторинга и контроля\\ \textbf{Ведущий специалист}]
		]
	]
	\node [draw, fit=(current bounding box.south east) (current bounding box.north west)] {};
\end{forest}
\caption{Организационная схема}
\end{figure}

Составим список событий, которые должны контролироваться в процессе мониторинга. Целесообразно рассмотреть таблицу MS Windows Server Security Log.

\begin{table}[H]
	\centering
	\caption{Список событий для мониторинга}
	\begin{tabular}{|l|c|}
		\hline
		\multicolumn{1}{|c|}{Событие}                      & \multicolumn{1}{c|}{Код} \\ \hline
		Вход с учётной записью выполнен успешно            & 4624                     \\ \hline
		Учётной записи не удалось выполнить вход в систему & 4625                     \\ \hline
		Попытка сбросить пароль учётной записи             & 4724                     \\ \hline
		Создана учётная запись пользователя                & 4720                     \\ \hline
		Отключена учётная запись пользователя              & 4725                     \\ \hline
		Изменена учётная запись пользователя               & 4738                     \\ \hline
		Заблокирована учётная запись пользователя          & 4740                     \\ \hline
		Удалена учётная запись пользователя                & 4726                     \\ \hline
		Создана группа пользователей                       & 4731                     \\ \hline
		Изменена группа пользователей                      & 4735                     \\ \hline
		Удалена группа пользователей                       & 4734                     \\ \hline
		В группу пользователей добавлен участник           & 4732                     \\ \hline
		Из группы пользователей удален участник            & 4733                     \\ \hline
		Подключен СМНИ            &    Назначается политикой ИБ                  \\ \hline		
		Извлечен СМНИ            &   Назначается политикой ИБ                   \\ \hline				
		Антивирус запущен            &   Назначается политикой ИБ                   \\ \hline		
		Антивирус остановлен            & Назначается политикой ИБ                      \\ \hline				
		Антивирус обновлен            & Назначается политикой ИБ                      \\ \hline						
	\end{tabular}
\end{table}

Попробуем описать требования к специалистам в рамках департамента ИБ.

\textbf{Начальник отдела} выполняет работы по технической защите информации в организациях. Руководит работами по выявлению угроз безопасности информации, определению возможностей технической разведки и проведению мероприятий технической защиты информации. Участвует в категорировании объектов информатизации, выявлении угроз безопасности информации и технических каналов утечки информации, работах по проведению специальных проверок и специальных исследований объектов информатизации. Формулирует цели и задачи работ по созданию безопасных информационных технологий, отвечающих требованиям технической защиты информации, определяет перспективы их развития для конкретных объектов защиты. Возглавляет разработку проектов перспективных и текущих планов работ, составление отчетов по их выполнению. Принимает участие в разработке технических заданий и проведении научных исследований, выполняемых в организации, а также обеспечивает взаимодействие соисполнителей по научно-исследовательским и опытно-конструкторским работам в части обеспечения технической защиты информации. Участвует в организации и осуществлении мероприятий по предотвращению утечки информации ограниченного доступа должностными лицами организаций, выполняющих работы, связанные со сведениями, составляющими государственную тайну и (или) содержащими иную информацию ограниченного доступа, при использовании открытых каналов радиосвязи. Руководит работами по составлению актов, протоколов испытаний, предписаний на право эксплуатации и другой документации по технической защите информации и обеспечению безопасности информации на объектах информатизации, доработке (модернизации) средств и систем технической защиты информации, их внедрению, информированию заинтересованных организаций о сущности и эффективности доработок, принимает меры по обеспечению финансирования этих работ, в том числе выполняемых по договорам. Определяет направления работы отдела (лаборатории, сектора) по разработке безопасных информационных технологий, согласовывает проектную и другую техническую документацию на вновь создаваемые и модернизируемые объекты в части выполнения требований по технической защите информации, обеспечивает высокий научно-технический уровень работ, эффективность и качество исследований и разработок. Оценивает потребность и организует обеспечение (снабжение) отдела (лаборатории, сектора) техникой и оборудованием, а также материальными, финансовыми и другими ресурсами, обеспечивая их рациональное расходование. Организует учет, использование по назначению, техническое обслуживание, ремонт, хранение и списание техники отдела и другого оборудования. Участвует в подборе, расстановке и аттестации специалистов. Руководит работниками отдела (лаборатории, сектора), принимает меры по их профессиональной переподготовке и повышению квалификации.

\textbf{Специалист по технической защите информации} выполняет работы по технической защите информации в организациях. Проводит работы по выявлению угроз безопасности информации, определению возможности технической разведки и проведению мероприятий технической защиты информации. Участвует в категорировании объектов информатизации, выявлении угроз безопасности информации и технических каналов утечки информации, работах по проведению специальных проверок и специальных исследований объектов информатизации. Разрабатывает предложения по размещению основных и вспомогательных технических средств и систем с соблюдением установленных норм технической защиты информации. Организует и проводит (при необходимости) мероприятия по технической защите информации при размещении сторонних организаций в пределах контролируемой зоны. Участвует в обследовании объектов информатизации, их категорировании и аттестации. Разрабатывает и готовит к утверждению проекты нормативных и методических документов, регламентирующих работу по технической защите информации, акты проверки, протоколы испытаний, предписания на право эксплуатации, а также положения, инструкции и другие организационно-распорядительные документы. Участвует в определении потребности в средствах технической защиты информации, составляет заявки на их приобретение с необходимыми обоснованиями и расчетами к ним, контролирует их поставку и использование. Осуществляет проверку выполнения требований нормативных документов по технической защите информации.

\textbf{Администратор ИБ} устанавливает разграничение полномочий пользователей и порядок доступа к информационным ресурсам, порядок использования основных и вспомогательных технических средств и систем. Проводит контроль выполнения работниками организации работ согласно перечню мероприятий по обеспечению безопасности информации, ведет учет нештатных ситуаций; информирует руководство и уполномоченных работников службы безопасности об инцидентах и попытках несанкционированного доступа к информации, элементам автоматизированных систем управления по результатам функционирования и контроля систем технической защиты информации. Осуществляет администрирование сервисами и механизмами безопасности автоматизированных систем управления, комплексами и средствами технической защиты информации и контроля; прекращает работы при несоблюдении установленной технологии обработки информации и невыполнении требований информационной безопасности; готовит предложения по совершенствованию технологических мер защиты информации. Контролирует работы по установке, модернизации и профилактике аппаратных и программных средств; созданию, учету, хранению и использованию резервных и архивных копий массивов данных и электронных документов. Принимает участие в работах по внесению изменений в программно-аппаратную конфигурацию автоматизированных систем управления и контролирует ее соответствие требованиям обеспечения безопасности информации. Ведет учет носителей информации, осуществляет их хранение, прием, выдачу ответственным исполнителям, контролирует правильность их использования

\textbf{Инженер по технической защите информации} выполняет работы по внедрению специальных технических и программно-математических средств защиты информации, обеспечению организационных и инженерно-технических мер защиты информационных систем. Разрабатывает планы и графики работ по техническому обслуживанию и ремонту электронного оборудования технических средств защиты информации и повышению эффективности его использования. Участвует в проверке, приемке и освоении вновь вводимого в эксплуатацию оборудования. Ведет учет неисправностей, поломок и аварий оборудования, анализирует причины и определяет направления их устранения; изучает режимы работы оборудования и условия его эксплуатации. Выполняет работы по эксплуатации, обслуживанию и ремонту средств технической защиты информации, разрабатывает предложения по доработке (модернизации) оборудования, повышающие его надежность, долговечность и эффективность применения. Составляет заявки на приобретение оборудования, запасного имущества, принадлежностей и материалов к нему, ремонт неисправных устройств. Организует хранение и списание оборудования и средств защиты информации, не подлежащих дальнейшему использованию по назначению.


Теперь, формализовав требования к специалистам в рамках департамента ИБ, и знаю задачи мониторинга, можно так их соотнести:

\begin{table}[H]
	\centering
	\begin{tabular}{|p{10cm}|p{4cm}|}
		\hline
		\textbf{Задача}                                                                                                                                             & \textbf{Специалисты}                \\ \hline
		Контроль за событиями безопасности и действиями пользователей                                                                                               & Администратор ИБ                   \\ \hline
		Контроль защищенности информации                                                                                                                            & Все специалисты                     \\ \hline
		Анализ и оценка функционирования системы защиты информации, включая выявление, анализ и устранение недостатков в функционировании системы защиты информации & Начальник отдела, специалист по ТЗИ \\ \hline
		Периодический анализ изменения угроз безопасности информации и принятие мер защиты информации в случае возникновения новых угроз безопасности информации    & Начальник отдела, специалист по ТЗИ \\ \hline
		Документирование процедур и результатов мониторинга за обеспечением уровня защищенности информации                                                          & Администратор ИБ, инженер по ТЗИ    \\ \hline
	\end{tabular}
\end{table}



