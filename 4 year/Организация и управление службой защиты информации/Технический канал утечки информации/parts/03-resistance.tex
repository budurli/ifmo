\section{Противодействие ТКУ}

Инженерно-техническая защита информации предполагает комплекс мероприятий по защите информации от несанкционированного доступа по различным каналам, а также нейтрализацию специальных воздействий на неё – уничтожения, искажения или блокирования доступа.

Концепция инженерно-технической защиты ин- формации определяет основные принципы, методы и средства обеспечения информационной без- опасности объектов. Она представляет собой общий замысел и принципы обеспечения информационной безопасности объекта в условиях угроз и включает в себя:

\begin{itemize}
	\item оценку угроз и ресурсы, подлежащие защите;
	\item систему защиты информации, принципы ее организации функционирования;
	\item принципы построения системы защиты информации;
	\item правовые основы.
\end{itemize}

Эффективная техническая защита информационных ресурсов является неотъемлемой частью комплексной системы обеспечения информационной безопасности и способствует оптимизации финансовых затрат на организацию защиты информации.

Эффективность системы защиты основных и вспомогательных технических средств от утечки информации по техническим каналам оценивается по различным критериям, которые определяются физической природой информационного сигнала, но чаще всего по соотношению «сигнал/шум».

Защита информации от утечки через ПЭМИН осуществляется с применением пассивных и активных методов и средств.

Цель пассивных и активных методов защиты – уменьшение отношения сигнал / шум (ОСШ) на границе контролируемой зоны до величин, обеспечивающих невозможность выделения средством разведки противника опасного информационного сигнала. В пассивных методах защиты уменьшение отношения С/Ш достигается путём уменьшения уровня опасного сигнала, в активных методах – путём увеличения уровня шума.

Пассивные методы защиты информации направлены на:
\begin{itemize}
	\item ослабление побочных электромагнитных излучений ОТСС на границе контролируемой зоны;
	\item  ослабление наводок побочных электромагнитных излучений в посторонних проводниках, соединительных линиях, цепях электропитания и заземления, выходящих за пределы контролируемой зоны;
	\item исключение или ослабление просачивания информационных сигналов в цепи электропитания и заземления, выходящие за пределы контролируемой зоны.
\end{itemize}

Ослабление опасного сигнала необходимо проводить до величин, обеспечивающих невозможность его выделения средством разведки на фоне естественных шумов.

\begin{figure}[H]
	\centering
	\includegraphics{"img/defence"}
	\caption{Классификация методов защиты}
\end{figure}


К пассивным методам защиты относятся:
\begin{itemize}
	\item применение разделительных трансформаторов и помехоподавляющих фильтров;
	\item экранирование;
	\item заземление всех устройств, как необходимое условие эффективной защиты информации;
	\item доработка устройств ВТ с целью минимизации уровня излучения.
\end{itemize}

Активные методы защиты информации направлены на:
\begin{itemize}
\item создание маскирующих пространственных электромагнитных помех;
\item создание маскирующих электромагнитных помех в посторонних проводниках, соединительных линиях, цепях электропитания и заземления.
\end{itemize}

К активным методам защиты относятся пространственное и линейное зашумление.