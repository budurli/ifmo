\section{Способы защиты}

Проверка задержки по времени может потенциально обнаружить атаку в определённых ситуациях. Например, при длительных вычислениях хеш-функций, которые выполняются в течение десятка секунд. Чтобы выявить потенциальные атаки, стороны проверяют расхождения во времени ответа. Предположим, что две стороны обычно затрачивают определённое количество времени для выполнения конкретной транзакции. Однако, если одна транзакция занимает аномальный период времени для достижения другой стороны, это может свидетельствовать о вмешательстве третьей стороны, вносящей дополнительную задержку в транзакцию.

Для обнаружения атаки <<человек посередине>> необходимо анализировать сетевой трафик.

\subsection{Защита от ARP-spoofing}

Программа \textbf{arpwatch} отслеживает всю ARP-активность на указанных интерфейсах. Когда она замечает аномалии, например, изменение MAC-адреса при сохранении IP-адреса, или наоборот, она сообщает об этом в syslog.

Можно бороться со слабостями протокола ARP кардинально - просто не использовать его. ARP-таблицу можно сформировать вручную, при этом она становится неуязвимой к ARP-атакам. Для этого нужно добавить необходимые MAC-адреса в таблицу.

Если при этом отключить использование ARP на сетевых интерфейсах, то доступны будут только те системы, MAC-адреса которых добавлены в ARP-таблицу нашего узла и наш MAC-адрес добавлен в ARP-таблицы узлов, с которыми производится обмен трафиком.

\subsection{DHCP-snooping}
DHCP snooping — функция коммутатора, предназначенная для защиты от атак с использованием протокола DHCP. Например, атаки с подменой DHCP-сервера в сети или атаки DHCP starvation, которая заставляет
DHCP-сервер выдать все существующие на сервере адреса злоумышленнику.

DHCP snooping регулирует только сообщения DHCP и не может повлиять напрямую на трафик пользователей или другие протоколы

По умолчанию коммутатор отбрасывает DHCP-пакет, который пришел на ненадёжный порт, если:
\begin{itemize}
	\item Приходит одно из сообщений, которые отправляет DHCP-сервер (DHCPOFFER, DHCPACK, DHCPNAK или DHCPLEASEQUERY);
	\item приходит сообщение DHCPRELEASE или DHCPDECLINE, в котором содержится MAC-адрес из базы данных привязки DHCP, но информация об интерфейсе в таблице не совпадает с интерфейсом, на котором был получен пакет;
	\item в пришедшем DHCP-пакете не совпадают MAC-адрес, указанный в DHCP-запросе, и MAC-адрес отправителя;
	\item приходит DHCP-пакет, в котором есть опция 82.
\end{itemize}
